\documentclass[12pt]{article}
\usepackage{amsmath, chemformula, booktabs, siunitx, tikz, ctex}
\usetikzlibrary{positioning}

\title{核反应实验分析}
\author{中国原子能科学研究院}

\begin{document}

\maketitle

\section*{实验数据表}
\begin{table}[h]
\centering
\caption{\ce{^{19}F(p,\alpha)^{16}O} 反应实验数据}
\begin{tabular}{ccc}
\toprule
质子共振能量 (Lab) & $\gamma$能量 & $\alpha$能量 (质心系) \\
(\si{keV}) & (\si{MeV}) & (\si{MeV}) \\
\midrule
668 & 6.13 & 1.30 \\
843 & 无 & 7.14 \\
874 & 6.13 & 1.46 \\
\bottomrule
\end{tabular}
\end{table}

\section*{能级示意图}
\begin{center}
\begin{tikzpicture}[
    level/.style={thick, minimum width=2cm},
    arrow/.style={->, >=stealth, thick}
]

% ²⁰Ne 能级
\node[level,fill=blue!20] (ne-base) at (0,0) {\ce{^{20}Ne} 基态};
\node[level,fill=red!20] (ne-excited) at (0,8.9) {\ce{^{20}Ne^*} (\SI{8.909}{MeV})};

% ¹⁶O 能级
\node[level,fill=green!20] (o-base) at (5,0) {\ce{^{16}O} 基态};
\node[level,fill=yellow!20] (o-excited) at (5,6.13) {\ce{^{16}O^*} (\SI{6.13}{MeV})};

% 能量跃迁
\draw[arrow, red] (ne-excited.south) -- node[right] {$\alpha$ (\SI{7.14}{MeV})} (o-base.west);
\draw[arrow, blue, dashed] (ne-excited.east) -- node[above] {$\gamma$ (未观测)} (o-excited.west);

% 能量标尺
\draw[<->] (-1,0) -- node[left] {\SI{8.909}{MeV}} (-1,8.9);
\draw[<->] (6,0) -- node[right] {\SI{6.13}{MeV}} (6,6.13);

\end{tikzpicture}
\end{center}

\section*{理论解释}
在 \SI{843}{keV} 的(p,$\alpha$)共振反应中未观测到$\gamma$射线的原因:

\paragraph{1. 能量守恒分析}
\begin{align*}
Q &= \SI{8.11}{MeV} \\
E_{\text{质心}} &= 843 \times \frac{19}{20} = \SI{799}{keV} \\
E_{\text{激发态}} &= Q + E_{\text{质心}} = \SI{8.909}{MeV}
\end{align*}

\paragraph{2. 退激路径选择}
\begin{itemize}
\item 激发态直接通过$\alpha$衰变退激至基态
\item 理论$\gamma$能量应满足:
$$
E_\gamma = 8.909 - 6.13 = \SI{2.779}{MeV}
$$
但实验未观测到该能级跃迁
\end{itemize}

\paragraph{3. 选择定则限制}
\begin{itemize}
\item 激发态角动量 $J^\pi = 1^-$
\item 基态角动量 $J^\pi = 0^+$
\item $\alpha$粒子携带角动量 $L=1$ 满足守恒
\item $\gamma$跃迁需要 $\Delta L=1$ 但中间态无法满足
\end{itemize}

\paragraph{4. 寿命差异}
\begin{align*}
\tau_\alpha &\sim \SI{e-21}{s} \\
\tau_\gamma &\sim \SI{e-12}{s} \\
\frac{\lambda_\alpha}{\lambda_\gamma} &\approx 10^9
\end{align*}
$\alpha$衰变速率远高于$\gamma$跃迁速率

\section*{已知条件}
\begin{itemize}
  \item 氚半衰期:$ T_{1/2} = \SI{12.33}{年} $
  \item 氚气体积:$ V = \SI{1}{mL} = \SI{1e-3}{L} $
  \item 标准状态:$ T = \SI{273}{K}, P = \SI{1}{atm} $
  \item 理想气体常数:$ R = \SI{0.0821}{L\cdot atm/(mol\cdot K)} $
  \item 阿伏伽德罗常数:$ N_A = \SI{6.022e23}{mol^{-1}} $
\end{itemize}

\section*{计算过程}
\subsection*{1. 计算氚原子数 $ N $}
\begin{align*}
  n &= \frac{PV}{RT} \\
    &= \frac{1 \times 1e-3}{0.0821 \times 273} \\
    &= \SI{4.46e-5}{mol} \\
  N &= n \times N_A \\
    &= 4.46e-5 \times 6.022e23 \\
    &= \SI{2.68e19}{}
\end{align*}

\subsection*{2. 计算衰变常数 $ \lambda $}
\begin{align*}
  \lambda &= \frac{\ln 2}{T_{1/2}} \\
          &= \frac{0.693}{12.33 \times 3.154e7} \\
          &= \SI{1.78e-9}{s^{-1}}
\end{align*}

\subsection*{3. 计算β粒子发射强度 $ A $}
\begin{align*}
  A &= \lambda N \\
    &= 1.78e-9 \times 2.68e19 \\
    &= \SI{4.77e10}{Bq} \\
    &= \SI{1.29}{Ci} \quad (\because 1\ \text{居里} = \SI{3.7e10}{Bq})
\end{align*}

\section*{关键公式验证}
\begin{itemize}
  \item \textbf{理想气体定律}:验证在标准状态下计算的合理性
  $$
  \frac{PV}{nRT} = 1 \Rightarrow \frac{1 \times 1e-3}{4.46e-5 \times 0.0821 \times 273} = 1.002
  $$
  
  \item \textbf{半衰期转换误差}:验证时间单位转换精度
  $$
  1\ \text{年} = \SI{31556952}{s} \ (\text{回归年精确值}) \Rightarrow \lambda = \SI{1.7802e-9}{s^{-1}}
  $$
  
  \item \textbf{有效数字规则}:最终结果保留3位有效数字
  $$
  \frac{2.68e19}{1.76e21} \times 100\% = 1.52\% \ (\text{误差范围})
  $$
\end{itemize}

\end{document}