\documentclass{article}
\usepackage{amsmath}
\usepackage{ctex}
\usepackage{array}
\usepackage{siunitx}
\usepackage[top=1cm, bottom=1cm, left=2cm, right=2cm]{geometry}

\title{中国原子能科学研究院2007年博士考试解答}
\date{}

\begin{document}
\maketitle

\section*{一、问答题解答}
\begin{enumerate}
  \item \textbf{壳模型相关问题}
  \begin{enumerate}
    \item 主要实验依据:
    \begin{itemize}
      \item 原子核的幻数存在(2,8,20,28,50,82,126)
      \item 同核异能素岛现象
      \item 核磁矩实验值与单粒子模型预测相符
    \end{itemize}
    
    \item 基本思想:
    \begin{itemize}
      \item 核子在平均势场中独立运动
      \item 考虑自旋-轨道耦合作用
      \item 核子填充量子态遵循泡利原理
    \end{itemize}
    
    \item 基态自旋宇称:
    \begin{align*}
      ^{9}\text{Be} & : \frac{1}{2}^+ \quad (\text{1p}_{3/2} \text{空穴}) \\
      ^{14}\text{N} & : 1^+ \quad (\text{1p}_{1/2} \text{配对}) \\
      ^{37}\text{Cl} & : \frac{3}{2}^+ \quad (\text{2s}_{1/2} \text{单粒子})
    \end{align*}
  \end{enumerate}

  \item \textbf{标准模型与衰变相互作用}
  \begin{itemize}
    \item 四种基本相互作用:强、电磁、弱、引力
    \item $\alpha$衰变:强相互作用(形成α粒子)+ 电磁(隧穿效应)
    \item $\beta$衰变:弱相互作用
    \item $\gamma$衰变:电磁相互作用
  \end{itemize}

  \item \textbf{$\beta$与$\gamma$相互作用}
  \begin{enumerate}
    \item $\beta$衰变特点:
    \begin{itemize}
      \item 连续能谱(违反能量守恒→中微子假说)
      \item 类型:$\beta^-$(n→p)、$\beta^+$(p→n)、EC(p捕获e)
    \end{itemize}
    
    \item $\gamma$相互作用:
    \begin{itemize}
      \item 光电效应(低能)
      \item 康普顿散射(中能)
      \item 电子对效应(高能,$E_\gamma > 1.022$ MeV)
    \end{itemize}
  \end{enumerate}

  \item \textbf{核反应阶段与守恒定律}
  \begin{itemize}
    \item 三个阶段:
    \begin{enumerate}
      \item 直接反应($\sim10^{-22}$s)
      \item 复合核形成($\sim10^{-16}$s)
      \item 统计衰变($\sim10^{-12}$s)
    \end{enumerate}
    
    \item 守恒定律:
    \begin{itemize}
      \item 能量-动量守恒
      \item 角动量守恒
      \item 宇称守恒(弱相互作用除外)
    \end{itemize}
  \end{itemize}

  \item \textbf{中子相关问题}
  \begin{enumerate}
    \item 中子性质:
    \begin{itemize}
      \item 电荷中性,质量$m_n = 939.565$ MeV/c²
      \item 自旋1/2,磁矩$-1.913\mu_N$
    \end{itemize}
    
    \item 探测原理:
    \begin{itemize}
      \item 核反应法:$^{10}$B(n,α)$^7$Li
      \item 探测器:BF₃正比管、锂玻璃闪烁体
    \end{itemize}
    
    \item 飞行时间计算:
    $$
      t = \frac{L}{v} = L \sqrt{\frac{m_n}{2E_{kin}}}
    $$
    代入不同能量得:
    \begin{align*}
      1 \text{eV} & : \SI{72.3}{\micro s} \\
      100 \text{eV} & : \SI{7.23}{\micro s} \\
      10 \text{keV} & : \SI{229}{\nano s} \\
      1 \text{MeV} & : \SI{7.23}{\nano s}
    \end{align*}
  \end{enumerate}
\end{enumerate}

\section*{二、计算题解答}
\begin{enumerate}
  \item \textbf{钍-232能级分析}
  \begin{itemize}
    \item 自旋宇称判定:
    \begin{align*}
      49.8 \text{keV} & : 2^+ \quad (\text{E2跃迁}) \\
      163 \text{keV} & : 4^+ \quad (\text{E2跃迁}) \\
      333 \text{keV} & : 6^+ \quad (\text{E3跃迁}) \\
      555 \text{keV} & : 8^+ \quad (\text{E4跃迁})
    \end{itemize}
    \item 多极性依据:角动量变化$\Delta L = |J_f - J_i|$
  \end{itemize}

  \item \textbf{锂-6反应质子能量}
  \begin{align*}
    Q &= (m(^6\text{Li}) + m(d) - m(^7\text{Li}) - m(p))c^2 \\
      &= (14.086 + 13.136 - 14.908 - 7.289) = 5.025 \text{MeV} \\
    E_{p}^{cm} &= \frac{m_t}{m_t + m_p}E_d + Q \\
    基态质子能量 & : \SI{8.21}{MeV} \\
    激发态质子能量 & : \SI{7.73}{MeV}
  \end{align*}

  \item \textbf{铋-211α衰变}
  \begin{enumerate}
    \item 库仑势垒:
    $$
      V_C = \frac{Z_1Z_2e^2}{4\pi\epsilon_0R} = \frac{83 \times 2 \times 1.44 \text{MeV·fm}}{1.45(211^{1/3} + 4^{1/3})} = \SI{25.3}{MeV}
    $$
    
    \item 激发态能量:
    $$
      E_{exc} = E_{\alpha_0} - E_{\alpha_1} = 6.621 - 6.274 = \SI{0.347}{MeV}
    $$
    
    \item 衰变纲图:
    \begin{center}
      \begin{tikzpicture}
        \draw[->] (0,0) -- node[right] {$\alpha_0$} (0,-3) node[below] {$^{207}\text{Tl}$ 基态};
        \draw[->] (2,0) -- node[right] {$\alpha_1$} (2,-2) node[below] {激发态 (\SI{0.347}{MeV})};
      \end{tikzpicture}
    \end{center}
  \end{enumerate}

  \item \textbf{中子计数计算}
  \begin{align*}
    \text{束流粒子数} & : \frac{9\text{nA} \times 100\text{h}}{3e \times 3.6e3} = 8.33e^{12} \\
    \text{靶核数} & : \frac{1\text{mg/cm}^2 \times 6e^{23}}{96\text{g/mol}} = 6.25e^{18} \\
    \text{反应数} & : 8.33e^{12} \times 6.25e^{18} \times 1e^{-3}\text{b} = 5.21e^{8} \\
    \text{中子计数} & : 5.21e^{8} \times 4 \times 1\% = 2.08e^{7}
  \end{align*}
\end{enumerate}

\end{document}