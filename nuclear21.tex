\documentclass{article}
\usepackage{amsmath}
\usepackage{siunitx}
\usepackage[version=4]{mhchem}
\usepackage{ctex}
\usepackage{tikz}
\sisetup{separate-uncertainty, exponent-product=\cdot}
\usetikzlibrary{arrows.meta, positioning, calc}
\title{核物理试题解答}
\author{}
\date{}

\begin{document}
\maketitle

\section*{一、名词解释(20分,每题2分)}
\begin{enumerate}
  \item \textbf{半衰期}:p24放射性核素数目减少到原来一半所需的时间,$T_{1/2} = \dfrac{\ln 2}{\lambda}$
  
  \item \textbf{对关联}:p198原子核中核子配对运动的现象,导致奇偶质量差和转动惯量变化
  
  \item \textbf{同位素}:p7质子数相同中子数不同的核素(如\ce{^{12}C}与\ce{^{14}C})
  
  \item \textbf{幻数}:p190使原子核特别稳定的核子数(2,8,20,28,50,82,126)
  
  \item \textbf{穆斯堡尔效应}:p182原子核无反冲发射/吸收γ射线的共振现象
  
  \item \textbf{镜像核}:p7质子数与中子数互换的核(如\ce{^{3}H}与\ce{^{3}He})
  
  \item \textbf{同位旋}:p19 为了描述核子的质子态和中子态引入的算符,量子数t=1/2,质子态的t3=+1/2,中子态的t3=-1/2
  
  \item \textbf{超容许跃迁}:p152 母核与子核的波函数很相像的跃迁
  
  \item \textbf{内转换电子}:p172 跃迁时把核的激发能直接传递给核外电子而发射出来的电子
  
  \item \textbf{同核异能素}:p7,p175 质量数和质子数均相同,而能量状态不同的核素
\end{enumerate}

\section*{二、简述题(16分,每题4分)}
\begin{enumerate}
  \item \textbf{γ射线与物质相互作用p64}:
  \begin{itemize}
    \item 光电效应(低能γ被原子整体吸收)
    \item 康普顿散射(γ与电子非弹性碰撞)
    \item 电子对效应($\gamma \to e^+ + e^-$,需$E_\gamma > 1.022$ MeV)
  \end{itemize}

  \item \textbf{β能谱特点与中微子假说p129,191}:
  \begin{itemize}
    \item 连续能谱(传统理论无法解释)
    \item 泡利提出中微子假说:$\beta$衰变能量由$e^-$、$\nu$、子核三者分配
    \item 实验测得$E_{\text{max}} = Q_\beta$,验证假说
  \end{itemize}

  \item \textbf{细致平衡原理p248}:
  \begin{itemize}
    \item 微观可逆性:$\sigma_{a\to b} = \sigma_{b\to a} \times \dfrac{p_b^2}{p_a^2}$
    \item 适用范围:时间反演对称系统,无极化观测时成立
  \end{itemize}

  \item \textbf{β衰变选择定则}:
  \begin{itemize}
    \item 容许跃迁选择定则:$\Delta I=0, \pm 1$,$\Delta\pi=+1$
    \item 其中0→0是纯费米跃迁,$\Delta I=\pm 1$是纯Gamow-Teller跃迁,$\Delta I=0$(除0→0)是F跃迁和G-T跃迁的混合
  \end{itemize}
\end{enumerate}

\section*{三、计算题(64分)}

\subsection*{第1题:\ce{^{7}Li}激发能计算}
\textbf{教材p233原题}:


\subsection*{第2题:\ce{^{198}Au}放射性活度计算}
\textbf{已知p24,242,300,9-10}:
\begin{itemize}
  \item 金箔厚度:\SI{0.02}{cm}
  \item 中子通量:\SI{1e12}{cm^{-2}.s^{-1}}
  \item 截面:\SI{98.7}{b} = \SI{98.7e-24}{cm^2}
  \item 照射时间:5分钟 = \SI{300}{s}
  \item 金密度:\SI{19.3}{g/cm^3}
  \item $T_{1/2} = 2.7\ \text{天} = 233280\ \si{s}$
\end{itemize}

\textbf{解}:
\begin{enumerate}
  \item 计算金原子数密度:
  $$
  n = \frac{\rho N_A}{A} = \frac{19.3 \times 6.022e23}{197} = \SI{5.89e22}{cm^{-3}}
  $$
  
  \item 核反应数:
  $$
  N = \phi \sigma n t d = 1e12 \times 98.7e-24 \times 5.89e22 \times 300 \times 0.02 = \SI{3.49e13}{}
  $$
  
  \item 放射性活度:
  $$
  A = \lambda N = \frac{\ln 2}{233280} \times 3.49e13 = \SI{1.04e8}{Bq/cm^2}
  $$
\end{enumerate}

\subsection*{第3题:\ce{^{239}Pu}α衰变计算}
\textbf{已知}:
\begin{itemize}
  \item $m_{\ce{^{239}Pu}} = \SI{0.5}{kg}$
  \item $E_\alpha = \SI{5.1}{MeV}$
  \item $T_{1/2} = 2.39e4\ \text{年} = 7.54e11\ \si{s}$
  \item $1u = \SI{1.66e-24}{g}$
\end{itemize}

\textbf{解}:
\begin{enumerate}
  \item 衰变方程:
  $$
  \ce{^{239}_{94}Pu -> ^{235}_{92}U + }\alpha
  $$

  \item 衰变能:
  $$
  E_d = \frac{A}{A-4}E_k = \frac{239}{235} \times 5.1 MeV = 5.187 MeV
  $$
  
  \item 衰变常数:
  $$
  \lambda = \frac{\ln 2}{T_{1/2}} = \frac{0.693}{7.54e11} = \SI{9.19e-13}{s^{-1}}
  $$
  
  \item 计算原子数:
  $$
  N = \frac{500}{239} \times N_A = \SI{1.26e24}{}
  $$
  
  \item 衰变率:
  $$
  A = \lambda N = 9.19e-13 \times 1.26e24 = \SI{1.16e12}{Bq}
  $$
  
  \item 功率计算:
  $$
  P = A \times E_k \times 1.6e-13 = 1.16e12 \times 5.1 \times 1.6e-13 = \SI{0.095}{W}
  $$
\end{enumerate}






\section*{四、中子分离能计算}
\textbf{题目:}从\ce{^{16}O}和\ce{^{17}O}核中取出一个中子,各需多少能量?解释两者差异原因。

\textbf{已知:}
$$
\begin{aligned}
\Delta(n) &= 8.071\ \si{MeV} \\
\Delta(\ce{^{15}O}) &= 2.856\ \si{MeV} \\
\Delta(\ce{^{16}O}) &= -4.737\ \si{MeV} \\
\Delta(\ce{^{17}O}) &= -0.809\ \si{MeV}
\end{aligned}
$$

\textbf{解:}
\begin{enumerate}
  \item \ce{^{16}O}分离中子:
  $$
  S_n(\ce{^{16}O}) = \Delta(\ce{^{15}O}) + \Delta(n) - \Delta(\ce{^{16}O}) = 2.856 + 8.071 - (-4.737) = \SI{15.664}{MeV}
  $$
  
  \item \ce{^{17}O}分离中子:
  $$
  S_n(\ce{^{17}O}) = \Delta(\ce{^{16}O}) + \Delta(n) - \Delta(\ce{^{17}O}) = (-4.737) + 8.071 - (-0.809) = \SI{4.143}{MeV}
  $$
  
  \item 差异原因:
  \begin{itemize}
    \item \ce{^{16}O}是双幻数核($Z=8,N=8$),壳层填充完整,结合能大
    \item \ce{^{17}O}多一个中子处于未填满壳层,结合较松
  \end{itemize}
\end{enumerate}

\section*{五、铀核性质计算}
\textbf{题目:}计算\ce{^{238}U}的核半径与核物质密度($r_0=1.45\ \si{fm}$)

\textbf{解:}
\begin{enumerate}
  \item 核半径:
  $$
  R = r_0 A^{1/3} = 1.45 \times 238^{1/3} \approx \SI{8.99}{fm}
  $$
  
  \item 核密度:
  $$
  \rho = \frac{m}{V} = \frac{238 \times (1.66 \times 10^{-24}\ \si{\gram})}{\frac{4}{3}\pi \left(8.99 \times 10^{-13}\ \si{\centi\meter}\right)^3} \approx \SI{1.3e14}{\gram\per\cubic\centi\meter}
  $$
\end{enumerate}

\section*{六、中子质量计算}
\textbf{题目:}利用\ce{^{210}Po}的α衰变产生α粒子进行核反应,求中子质量。

\textbf{已知:}
$$
\begin{aligned}
\Delta(\ce{^{210}Po}) &= -15.969\ \si{MeV} \\
\Delta(\ce{^{206}Pb}) &= -23.081\ \si{MeV} \\
\Delta(\alpha) &= 2.425\ \si{MeV} \\
\Delta(\ce{^{11}B}) &= 8.668\ \si{MeV} \\
\Delta(\ce{^{14}N}) &= 2.863\ \si{MeV} \\
K_N &= 0.8\ \si{MeV},\quad K_n = 4.31\ \si{MeV}
\end{aligned}
$$

\textbf{解:}
\begin{enumerate}
  \item α衰变Q值:
  $$
  Q_\alpha = \Delta(\ce{^{210}Po}) - [\Delta(\ce{^{206}Pb}) + \Delta(\alpha)] = 4.687\ \si{MeV}
  $$
  
  \item α粒子动能:
  $$
  K_\alpha = \frac{206}{206+4}Q_\alpha \approx \SI{4.599}{MeV}
  $$
  
  \item 反应Q值方程:
  $$
  Q = (\Delta(\ce{^{11}B}) + \Delta(\alpha)) - (\Delta(\ce{^{14}N}) + \Delta(n)) = 8.23 - \Delta(n)
  $$
  
  \item 能量守恒:
  $$
  Q + K_\alpha = K_N + K_n \Rightarrow \Delta(n) = \SI{7.719}{MeV}
  $$
  
  \item 中子质量:
  $$
  m_n = 1u + \frac{7.719}{931.494} \approx \SI{1.00828}{u}
  $$
\end{enumerate}

\section*{七、衰变纲图绘制}
P139 图6-11
% \begin{center}
% \begin{tikzpicture}[
%     scale=0.8,
%     transform shape,
%     every node/.style={align=center},
%     decay/.style={->, >=Stealth, thick},
%     energy level/.style={thick, minimum width=1.5cm}
%   ]
  
%   % ====== 能级定义 ======
%   % 母核
%   \node[energy level] (Cu) at (-4,0) {\ce{^{64}_{29}Cu}\\$1^+$};
  
%   % 子核能级
%   \node[energy level] (Zn) at (4,0)  {\ce{^{64}_{30}Zn}\\$0^+$};
%   \node[energy level] (NiStar) at (4,3) {\ce{^{64}_{28}Ni^*}\\$2^+$};
%   \node[energy level] (Ni) at (4,-3) {\ce{^{64}_{28}Ni}\\$0^+$};
  
%   % ====== 衰变路径 ======
%   % β-衰变到Zn
%   \draw[decay, red] (Cu) 
%     to[out=0, in=180] 
%     node[above, pos=0.7] {$\beta^-$\\0.573 MeV\\40\%} 
%     (Zn);
  
%   % β+衰变到Ni基态
%   \draw[decay, blue] (Cu) 
%     to[out=340, in=200] 
%     node[below, pos=0.7] {$\beta^+$\\0.654 MeV\\19\%} 
%     (Ni);
  
%   % 电子俘获到Ni基态
%   \draw[decay, green!50!black, dashed] (Cu) 
%     to[out=315, in=180] 
%     node[above left, pos=0.6] {EC\\40.4\%} 
%     (Ni.north west);
  
%   % 电子俘获到激发态
%   \draw[decay, orange, dashed] (Cu) 
%     to[out=45, in=180] 
%     node[above right, pos=0.6] {EC\\0.6\%} 
%     (NiStar.west);
  
%   % γ跃迁
%   \draw[decay, purple] (NiStar) 
%     -- 
%     node[right] {$\gamma$\\1.348 MeV} 
%     (Zn);

%   % ====== 标注框 ======
%   \node[draw=gray, rounded corners, text width=8cm, align=left, anchor=north west] 
%     at (-6,-5) {
%     \textbf{关键特征量:}
%     \begin{itemize}
%       \item 半衰期:$T_{1/2} = 12.7\ \text{小时}$
%       \item 分支比总和:$40\% + 19\% + 40.4\% + 0.6\% = 100\%$
%       \item 激发态寿命:$\tau \sim 10^{-12}\ \text{秒}$
%       \item $Q_\beta = 1.348\ \text{MeV}$
%     \end{itemize}
%   };

% \end{tikzpicture}
% \end{center}


\end{document}