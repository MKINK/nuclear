\documentclass[12pt]{article}
\usepackage{amsmath, chemformula, booktabs, siunitx, tikz, ctex}
\usetikzlibrary{positioning}

\title{核反应实验分析}
\author{中国原子能科学研究院}

\begin{document}

\maketitle

\section*{题目:p+$ \ce{^{19}F} $ 实验结果分析}
\begin{table}[h]
\centering
\caption{p+$ \ce{^{19}F} $ 反应实验数据}
\begin{tabular}{ccc}
\toprule
质子共振能量 (Lab) & $\gamma$能量 & $\alpha$能量 (质心系) \\
(\si{keV}) & (\si{MeV}) & (\si{MeV}) \\
\midrule
668 & 6.13 & 1.30 \\
668 & 6.92 & 1.47 \\
668 & 7.12 & 2.10 \\
843 & 无 & 7.14 \\
874 & 6.13 & 1.46 \\
874 & 6.92 & 1.62 \\
874 & 7.12 & 2.25 \\
\bottomrule
\end{tabular}
\end{table}

已知 $ \ce{^{19}F(p,\alpha)^{16}O} $ 反应的 $ Q = \SI{8.11}{MeV} $,要求:
\begin{enumerate}
  \item 绘制 $ \ce{^{20}Ne} $ 和 $ \ce{^{16}O} $ 的能级图
  \item 解释 843 keV 的 (p,$\alpha$) 共振反应中为何未观测到 $\gamma$ 射线
\end{enumerate}

\section*{解答}
\subsection*{1. 能级图绘制}
\begin{center}
\begin{tikzpicture}[
    level/.style={thick, minimum width=3cm, fill=blue!10},
    label/.style={midway, font=\small}
]
% ²⁰Ne 能级
\draw[level] (0,0) -- node[above] {基态} (4,0);
\draw[level] (0,8.74) -- node[above] {激发态 (8.74 MeV)} (4,8.74);  % 843 keV 对应态

% ¹⁶O 能级
\draw[level] (6,0) -- node[above] {基态} (10,0);
\draw[level] (6,6.13) -- node[above] {6.13 MeV} (10,6.13);
\draw[level] (6,7.12) -- node[above] {7.12 MeV} (10,7.12);

% 反应路径
\draw[->, thick, red] (2,8.74) -- node[label, left] {$\alpha$ (7.14 MeV)} (2,0);
\draw[->, thick, blue, dashed] (2,8.74) -- (8,6.13) node[label, above] {理论$\gamma$路径};
\end{tikzpicture}
\end{center}

\subsection*{2. $\gamma$ 射线缺失解释}
在 843 keV 共振反应中,未观测到 $\gamma$ 射线的原因如下:

\paragraph{能量守恒分析}
\begin{align*}
E_{\text{质心}} &= 843 \times \frac{19}{20} = \SI{799}{keV} \\
E_{\text{激发态}} &= Q + E_{\text{质心}} = 8.11 + 0.799 = \SI{8.909}{MeV}
\end{align*}

\paragraph{关键机制}
\begin{itemize}
  \item \textbf{直接衰变优先}:激发态寿命 $ \tau \sim \SI{e-21}{s} $,$\alpha$ 衰变速率 ($ \lambda_\alpha \sim \SI{e21}{s^{-1}} $) 远高于 $\gamma$ 跃迁速率 ($ \lambda_\gamma \sim \SI{e12}{s^{-1}} $)
  
  \item \textbf{能级不匹配}:
  $$
  \Delta E = 8.909 - 6.13 = \SI{2.779}{MeV} \quad (\text{无对应实验$\gamma$能量观测})
  $$
  
  \item \textbf{角动量选择定则}:
  \begin{align*}
  J^{\pi}(\ce{^{20}Ne^*}) &= 1^- \\
  J^{\pi}(\ce{^{16}O}) &= 0^+ \\
  \Delta L &= 1 \ (\text{需要} \ E1 \ \text{跃迁}) \quad \text{但中间态无法满足角动量守恒}
  \end{align*}
\end{itemize}

\paragraph{实验验证}
通过 $\alpha$ 粒子能量反推:
$$
E_{\alpha}^{\text{理论}} = 8.909 - 0 = \SI{8.909}{MeV} \quad (\text{实测} \ 7.14\ \si{MeV},\ \text{动量分配导致差异})
$$

\section*{题目:试写出两种可利用的聚变方程式}
以下是两种常见的可利用的核聚变反应方程式:

1. **氘-氚(D-T)聚变反应**  
   \[
   \ce{^2_1H + ^3_1H -> ^4_2He + ^1_0n + 17.6\ MeV}
   \]  
   **特点**:  
   • 释放能量高(17.6 MeV),反应条件相对较低(约1亿摄氏度)。  
   • 应用于国际热核实验堆(ITER)等聚变装置。  

2. **氘-氘(D-D)聚变反应(路径一)**  
   \[
   \ce{^2_1H + ^2_1H -> ^3_2He + ^1_0n + 3.27\ MeV}
   \]  
   **或路径二**:  
   \[
   \ce{^2_1H + ^2_1H -> ^3_1H + ^1_1H + 4.03\ MeV}
   \]  
   **特点**:  
   • 燃料易获取(氘存在于海水中),但需要更高温度(约4亿摄氏度)。  
   • 两种分支反应概率相近,总能量约3.65 MeV/反应。  

---

**注**:实际应用中,氘-氚反应因能量优势更为常用,而氘-氦3(\(\ce{^3_2He}\))反应因无中子辐射也被探索,但需极高温度(约10亿℃)。

以下是根据实验数据整理的题目和解答:

---

\section*{8}

#### **实验反应过程**
1. **质子诱发反应**  
\[
\ce{^{63}Cu + p -> ^{64}Zn^* -> \begin{cases} 
^{62}Zn + 2n \\ 
^{62}Cu + p + n 
\end{cases}}
\]
2. **α粒子诱发反应**  
\[
\ce{^{60}Ni + \alpha -> ^{64}Zn^* -> \begin{cases} 
^{62}Zn + 2n \\ 
^{62}Cu + p + n 
\end{cases}}
\]

#### **实验结果**
不同反应通道的截面比值为:  
\[
\frac{\sigma(p,2n)}{\sigma(p,pn)} : \frac{\sigma(\alpha,2n)}{\sigma(\alpha,pn)} = 1 : 1
\]

---

### **理论解答**

#### **1. 复合核模型理论**
根据Bohr的复合核假设,核反应分为两个独立阶段:  
• **复合核形成**:入射粒子与靶核融合形成激发态复合核(\(\ce{^{64}Zn^*}\))  
• **统计衰变**:复合核通过不同通道独立衰变  

反应截面可表示为:  
\[
\sigma(a,b) = \sigma_{\text{形成}}(a) \cdot P_{\text{衰变}}(b)
\]  
其中:  
• \(\sigma_{\text{形成}}(a)\):入射粒子\(a\)形成复合核的概率  
• \(P_{\text{衰变}}(b)\):复合核通过通道\(b\)衰变的概率  

---

#### **2. 截面比值分析**
对质子诱发和α诱发反应:  
\[
\frac{\sigma(p,2n)}{\sigma(p,pn)} = \frac{\sigma_{\text{形成}}(p) \cdot P(2n)}{\sigma_{\text{形成}}(p) \cdot P(pn)} = \frac{P(2n)}{P(pn)}  
\]  
\[
\frac{\sigma(\alpha,2n)}{\sigma(\alpha,pn)} = \frac{\sigma_{\text{形成}}(\alpha) \cdot P(2n)}{\sigma_{\text{形成}}(\alpha) \cdot P(pn)} = \frac{P(2n)}{P(pn)}  
\]  

由于两比值均退化为相同的衰变概率比,故有:  
\[
\frac{\sigma(p,2n)}{\sigma(p,pn)} = \frac{\sigma(\alpha,2n)}{\sigma(\alpha,pn)} = \frac{P(2n)}{P(pn)}
\]

---

#### **3. 衰变概率计算**
根据统计模型,衰变概率由能级密度决定:  
\[
P(b) \propto \frac{\rho(E^*,J^\pi)}{\sum \rho_i(E^*,J^\pi)}
\]  
其中:  
• \(\rho\):剩余核的能级密度函数  
• \(E^*\):激发能,\(J^\pi\):角动量和宇称  

对于相同的复合核\(\ce{^{64}Zn^*}\),无论由何种入射粒子形成:  
\[
\frac{P(2n)}{P(pn)} = \frac{\rho(^{62}Zn)}{\rho(^{62}Cu)}
\]  
仅与最终产物的能级密度相关,与入射粒子无关  

---

#### **4. 实验验证**
通过能级密度公式计算:  
\[
\rho(E^*) \propto \exp\left(2\sqrt{aE^*}\right)
\]  
其中\(a\)为能级密度参数,对\(^{62}Zn\)和\(^{62}Cu\):  
\[
a(^{62}Zn) \approx a(^{62}Cu) \quad (\text{因质量数相近})
\]  
得:  
\[
\frac{\rho(^{62}Zn)}{\rho(^{62}Cu)} \approx 1 \quad \Rightarrow \quad \frac{P(2n)}{P(pn)} \approx 1
\]  
与实验观测的截面比值一致  

---

### **结论**
实验结果表明:  
1. 不同入射粒子(\(p\)和\(\alpha\))形成的复合核\(\ce{^{64}Zn^*}\)具有相同的统计衰变特性  
2. 衰变通道的截面比值仅由终态核的能级密度决定,与复合核形成过程无关  
3. 实验结果验证了复合核模型的正确性  

---

**关键公式总结**  
\[
\boxed{
\frac{\sigma(p,2n)}{\sigma(p,pn)} = \frac{\sigma(\alpha,2n)}{\sigma(\alpha,pn)} = \frac{\rho(^{62}Zn)}{\rho(^{62}Cu)} \approx 1
}
\]


\end{document}