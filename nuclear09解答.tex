\documentclass{article}
\usepackage{amsmath}
\usepackage{enumitem}
\usepackage{textgreek}
\usepackage{ctex}
\begin{document}

\section*{2009年博士入学考试试题(CIAE)原子核实验方法}

\subsection*{一、选择题(每题5分)}

\begin{enumerate}[label=\arabic*.]
    \item 核反应$ ^2\text{H}(d, \gamma)^4\text{He} $产生的23.8\,MeV \textgamma\ 射线,穿过钯(Pd)层后经石蜡慢化,其可探测的最大能量为\underline{\hspace{2cm}}
    
    \textbf{答案:} 23.8\,MeV(γ射线的能量不会因慢化而改变)。
    
    \item 参与全部四种基本相互作用的粒子是:
    \begin{itemize}[label={},leftmargin=2em]
        \item A. \textbeta\ 粒子 
        \item B. \textgamma\ 光子
        \item C. \textpi\ 介子 
        \item D. \textupsilon\ 粒子
    \end{itemize}
    
    \textbf{答案:} C. \textpi\ 介子(π介子参与强、弱、电磁和引力相互作用)。
    
    \item \textbeta-衰变的母子核关系满足\underline{\hspace{2cm}}
    
    \textbf{答案:} 母子核的质量数相同,原子序数相差1。
    
    \item 某物质半衰期20天,128\,g该物质经过120天后剩余质量为\underline{\hspace{2cm}}
    
    \textbf{答案:} 剩余质量 \( m = 128 \times \left(\frac{1}{2}\right)^{120/20} = 128 \times \left(\frac{1}{2}\right)^6 = 2\, \text{g} \)。
    
    \item $^{189}\text{Os}$核半径$1/3$的稳定核为
    
    \textbf{答案:} $^{7}\text{Li}$(核半径与质量数的立方根成正比)。
    
    \item 核外电子总角动量$L=2$,自旋$S=0$,核$I=3/2$,则原子总角动量量子数可能值为:
    \begin{itemize}[label={},leftmargin=2em]
        \item A. $\frac{5}{2}, \frac{3}{2}, \frac{1}{2}, \frac{7}{2}$
        \item B. $-\frac{1}{2}, \frac{1}{2}, \frac{3}{2}, \frac{5}{2}, \frac{7}{2}$
        \item C. $\frac{5}{2}, \frac{7}{2}$ 
        \item D. $\frac{3}{2}, \frac{7}{2}$
    \end{itemize}
    
    \textbf{答案:} A. $\frac{5}{2}, \frac{3}{2}, \frac{1}{2}, \frac{7}{2}$(总角动量量子数为 \( J = |L \pm S| \))。
    
    \item 太阳能量来源的实质由什么产生:
    \begin{itemize}[label={},leftmargin=2em]
        \item A. $^1\text{H}$, $\alpha$, $\text{t}$
        \item B. $\alpha$, $^{12}\text{C}$,$^1\text{H}$ 
        \item C. $^1\text{H}$, $\text{d}$, $\text{t}$
        \item D. $^1\text{H}$
    \end{itemize}
    
    \textbf{答案:} D. $^1\text{H}$(太阳能量主要来自氢核聚变)。
    
    题目答案原图有点抽象,没看懂,太阳能量来自质子-质子链反应

    
    \item \textgamma-衰变中的内转换系数随:
    \begin{itemize}[label={},leftmargin=2em]
        \item A. 能量增大、原子序数增大而增大 
        \item B. 能量增大、原子序数减小而增大
        \item C. 能量减小、原子序数增大而增大 
        \item D. 能量减小、原子序数减小而增大
    \end{itemize}
    
    \textbf{答案:} C. 能量减小、原子序数增大而增大
\end{enumerate}

\subsection*{二、简答题(共15分)}

\begin{enumerate}[label=\arabic*.]
    \item (5分) 简述利用核反冲法测量核寿命的三种方法及原理
    
    \textbf{答案:}
    \begin{itemize}
        \item 反冲距离法:通过测量反冲核的飞行距离和时间计算寿命。
        \item 多普勒频移法:利用反冲核的多普勒频移效应测量寿命。
        \item 符合测量法:通过符合测量反冲核和衰变产物的时间差计算寿命。
    \end{itemize}
    
    \item (5分) 简述中子探测的原理,列举几种常用探测中子的气体探测器
    
    \textbf{答案:}
    \begin{itemize}
        \item 原理:中子与探测介质中的核发生反应,产生带电粒子或\(\gamma\)射线,通过探测次级粒子实现中子探测。
        \item 常用探测器:\( ^3\text{He} \) 正比计数器、BF\(_3\) 正比计数器、裂变室。
    \end{itemize}
    
    \item (5分) 已知$N_a=1000$,$N_b=2000$,求比值$R=\frac{N_a}{N_b}$的统计误差
    
    \textbf{答案:}
    统计误差为:
    \[
    \Delta R = R \sqrt{\left(\frac{\Delta N_a}{N_a}\right)^2 + \left(\frac{\Delta N_b}{N_b}\right)^2} = \frac{1000}{2000} \sqrt{\left(\frac{\sqrt{1000}}{1000}\right)^2 + \left(\frac{\sqrt{2000}}{2000}\right)^2} \approx 0.019
    \]
    
    \item (5分) 壳模型中$1p_{3/2}$轨道与$1p_{1/2}$能级劈裂的原因
    
    \textbf{答案:}
    自旋-轨道耦合作用导致能级劈裂,$1p_{3/2}$ 和 $1p_{1/2}$ 轨道的总角动量不同,能量不同。
    
    \item (5分) 试求$^{15}\text{O},^{17}\text{O}$核的$J^{\pi}$能级顺序
    
    \textbf{答案:}
    \begin{itemize}
        \item $^{15}\text{O}$:基态 \( J^\pi = \frac{1}{2}^- \),第一激发态 \( \frac{5}{2}^+ \)。
        \item $^{17}\text{O}$:基态 \( J^\pi = \frac{5}{2}^+ \),第一激发态 \( \frac{1}{2}^+ \)。
    \end{itemize}
    
    \item (5分) $^{18}\text{F}$可能的$J^{\pi}$组合
    
    \textbf{答案:}
    \( J^\pi = 1^+, 2^+, 3^+, 4^+, 5^+ \)(根据壳模型和奇偶核的组合)。
    
    \item (5分) 解释为何$^{18}\text{O}$核的自旋宇称为$J^{\pi}=0^+$
    
    \textbf{答案:}
    $^{18}\text{O}$ 核的质子数和中子数均为偶数,基态自旋为0,宇称为正。
\end{enumerate}

\subsection*{六、激发能计算(15分)}
$^{232}$Th核的低激发能为49.8\,keV、163\,keV、333\,keV、555\,keV。\\
求这几个激发能的$J^{\pi}$。

\textbf{答案:}
\begin{itemize}
    \item 49.8\,keV:\( J^\pi = 2^+ \)
    \item 163\,keV:\( J^\pi = 4^+ \)
    \item 333\,keV:\( J^\pi = 6^+ \)
    \item 555\,keV:\( J^\pi = 8^+ \)
\end{itemize}

\subsection*{七、核反应能量计算(20分)}
核反应$ ^6\text{Li} + d \rightarrow ^7\text{Li} + p $中,氘核结合能$B_d = 2\,\text{MeV}$。\\
在与入射方向垂直的位置观测到质子的两种能量分别为$a_1$和$a_2$,求$^7$Li第一激发态的激发能。 

参考p233例题

\textbf{答案:}
激发能 \( E^* = 8/7(a_1 - a_2) \)。

\subsection*{八、反应截面计算(20分)}
核反应$ A + B \rightarrow C + 4n $中:\\
\begin{itemize}
  \item 入射流强$I = 9\,\text{nA}$
  \item 靶B厚度$\rho = 1\,\text{mg/cm}^2$
  \item 靶核质量数$B = 96$
  \item 反应截面$\sigma = 1000\,\text{mb}$
  \item 探测器效率$\epsilon = 1\%$
\end{itemize}
求100小时内探测到的中子数。

\textbf{答案:}
要计算100小时内探测到的中子数,我们可以按照以下步骤进行:

1. **计算入射粒子数**:
   - 入射流强 \( I = 9 \, \text{nA} = 9 \times 10^{-9} \, \text{A} \)
   - 每个电子的电荷量 \( e = 1.6 \times 10^{-19} \, \text{C} \)
   - 入射粒子数率 \( R = \frac{I}{e} = \frac{9 \times 10^{-9}}{1.6 \times 10^{-19}} = 5.625 \times 10^{10} \, \text{粒子/秒} \)

2. **计算靶核数密度**:
   - 靶B厚度 \( \rho = 1 \, \text{mg/cm}^2 = 1 \times 10^{-3} \, \text{g/cm}^2 \)
   - 靶核质量数 \( B = 96 \)
   - 阿伏伽德罗常数 \( N_A = 6.022 \times 10^{23} \, \text{核/mol} \)
   - 靶核数密度 \( n = \frac{\rho \times N_A}{B} = \frac{1 \times 10^{-3} \times 6.022 \times 10^{23}}{96} \approx 6.273 \times 10^{18} \, \text{核/cm}^2 \)

3. **计算反应率**:
   - 反应截面 \( \sigma = 1000 \, \text{mb} = 1000 \times 10^{-27} \, \text{cm}^2 \)
   - 反应率 \( \lambda = R \times n \times \sigma = 5.625 \times 10^{10} \times 6.273 \times 10^{18} \times 1000 \times 10^{-27} \approx 3.528 \times 10^{5} \, \text{反应/秒} \)

4. **计算产生的中子数**:
   - 每次反应产生4个中子
   - 中子产生率 \( \lambda_n = 4 \times \lambda = 4 \times 3.528 \times 10^{5} \approx 1.411 \times 10^{6} \, \text{中子/秒} \)

5. **计算100小时内探测到的中子数**:
   - 探测器效率 \( \epsilon = 1\% = 0.01 \)
   - 100小时 \( = 100 \times 3600 = 3.6 \times 10^{5} \, \text{秒} \)
   - 探测到的中子数 \( N = \lambda_n \times \epsilon \times 3.6 \times 10^{5} = 1.411 \times 10^{6} \times 0.01 \times 3.6 \times 10^{5} \approx 5.080 \times 10^{9} \, \text{中子} \)

因此,100小时内探测到的中子数约为 \( 5.08 \times 10^{9} \) 个。

\subsection*{九、实验设计题(25分)}
\begin{enumerate}
  \item 设计实验验证如下衰变纲图:
  \begin{figure}[h]
    \centering
    % \includegraphics[width=0.8\textwidth]{decay_scheme}
    \caption{衰变纲图(含1.3\,MeV和2.1\,MeV两条$\beta$分支)}
  \end{figure}
  
  \item 要求:
  \begin{itemize}
    \item 画出实验装置方框图并说明各部件功能
    \item 详细描述符合测量原理
    \item 分析如何区分不同$\beta$衰变分支
  \end{itemize}
  
  \textbf{答案:}
  \begin{itemize}
    \item 实验装置:包括放射源、探测器、符合电路、数据采集系统。
    \item 符合测量原理:通过测量β射线和γ射线的符合事件,确定衰变分支。
    \item 区分β分支:通过能量分辨率和符合时间窗区分不同能量的β射线。
  \end{itemize}
\end{enumerate}

\end{document}