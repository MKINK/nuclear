\documentclass{article}
\usepackage{amsmath}
\usepackage{ctex}
\usepackage{array}
\usepackage[top=1cm, bottom=1cm, left=2cm, right=2cm]{geometry}

\title{中国原子能科学研究院\\2007年攻读博士学位研究生考试试题\\考试科目:原子核物理}
\date{共1页}

\begin{document}
\maketitle

\section*{注:}
\begin{itemize}
  \item[a)] 核素原子质量(质量过剩$\Delta$/MeV):
  \begin{center}
    \begin{tabular}{|c|c|c|c|c|c|c|c|}
      \hline
      元素符号 & n & $^1$H & $^2$H & $^3$He & $^{11}$B & $^{12}$C & $^{13}$C \\
      \hline
      $\Delta$ & 8.071 & 7.289 & 13.136 & 2.425 & 14.086 & 14.908 & 13.369 \\
      \hline
    \end{tabular}
  \end{center}

  \item[b)] 壳模型单粒子能级顺序:$1s_{1/2}/1p_{3/2}/1p_{1/2}/1d_{5/2}/2s_{1/2}/1d_{3/2}$;
  
  \item[c)] 基本参数:
  \begin{itemize}
    \item 电子静止质量 $m_ec^2=0.511$\,MeV
    \item 经典电子半径 $r_e=2.818\times10^{-6}$\,nm
    \item 1库仑$=6.24246\times10^{18}$\,e
    \item 阿伏伽德罗常量 $N_A=6\times10^{23}$\,mol$^{-1}$
    \item $\ln2=0.693$
  \end{itemize}

  \item[d)] 精确到小数点后3位:$\sqrt{3}/4=1.587$,$\sqrt{207}=5.915$
\end{itemize}

\section*{一、问答题(共50分)}
\begin{enumerate}
  \item 原子核模型中:
  \begin{enumerate}
    \item 壳模型的主要实验依据(3分)
    \item 壳模型的基本思想
    \item 根据壳模型,给出$^{9}$Be,$^{14}$N,$^{37}$Cl等核基态的自旋和宇称。
  \end{enumerate}

  \item 标准模型的四种基本相互作用是什么;原子核$\alpha$、$\beta$、$\gamma$衰变过程中分别包括哪些基本相互作用?(8分)

\item 
  \begin{enumerate}
    \item $\beta$衰变能谱的主要特点及三种$\beta$衰变类型(5分)
    \item $\gamma$射线与物质的三种主要相互作用(5分)
  \end{enumerate}

  \item 简述核反应过程的三个阶段?遵守哪几个主要守恒定律?(8分)

   \item 
  \begin{enumerate}
    \item 中子的主要性质
    \item 探测种子的基本原理及常用的中子探测器有哪些
    \item 常通过飞行时间法测量中子能谱,试计算1eV,100eV,10keV,1MeV中子飞行1m所需时间。
  \end{enumerate}
\end{enumerate}

\section*{二、计算题(50分)}
\begin{enumerate}
  \item 实验测得$_{90}^{232}$Th的激发态的能级能量依次为49.8keV,163keV,333keV,555keV,试求这些能级的自旋和宇称,并确定它们向下的$\gamma$跃迁的多级性。
  

  \item $^6$Li+d $\rightarrow$  $^7$Li+p,入射d束的能量为2.0MeV,在90°出射角方向放置带电粒子探测器测量两组分别对应$^7$Li处于基态和第一激发态(能量为0.478MeV)的质子,求这两组质子的能量

  \item $_{83}^{211}$Bi通过$\alpha$衰变至$_{81}^{207}$Tl
  \begin{enumerate}
      \item 计算该$\alpha$衰变的库伦势垒高度($r_0=1.45×10^{-13}cm$)
      \item 实验测得该衰变中有两组 \(\alpha\) 粒子,能量分别为 \(E(\alpha_0) = 6.621 \text{ MeV}\),\(E(\alpha_1) = 6.274 \text{ MeV}\),分别对应衰变到子核的基态和激发态。试求 \(_{81}^{207}\text{Tl}\) 激发态的能量,并画出此衰变纲图。
  \end{enumerate}

  \item 用流强为 \(9 \text{ nA}\),能量为 \(35 \text{ MeV}\) 的 \(^7\text{Li}\) 离子(电荷态 \(3+\))轰击 \(1 \text{ mg/cm}^2\) 的 \(^{96}\text{Zr}\) 靶,引起 \(^7\text{Li} + ^{96}\text{Zr} \rightarrow ^{99}\text{Tc} + 4n\) 反应的截面为 \(1000 \text{ 毫靶}\)。用总探测效率为 \(1\%\) 的中子探测器阵列探测该反应过程中产生的中子,问轰击 \(100\) 小时后中子探测器阵列记录该反应产生中子的总计数为多少?
\end{enumerate}


\end{document}





