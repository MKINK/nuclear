\documentclass{article}
\usepackage{graphicx} % Required for inserting images
\usepackage{ctex}
\usepackage{enumitem}
\usepackage{amsmath} 
\usepackage[a4paper, margin=1in]{geometry}
\title{English}
\author{Zhang}

\begin{document}

\maketitle

\section*{2020}
\subsection{Passage One 答案解析}


\subsubsection*{Question 1}
\textbf{题目}: 第一段中,作者暗示:\\
\textbf{选项}: 
\begin{enumerate}[label=\Alph*)]
    \item 人类前方道路充满障碍
    \item 我们相信人性本恶
    \item 人类被垃圾包围
    \item 从人的善良本性出发更能看清未来
\end{enumerate}
\textbf{答案}: \textbf{D} \\
\textbf{依据}:
\begin{itemize}
    \item 关键句:"something solid enough to step onto the look beyond the pile"(坚实的立足点以超越困境)
    \item 语义逻辑:强调从积极人性特质(good nature)出发是突破困境的关键
\end{itemize}

\subsubsection*{Question 2}
\textbf{题目}: 根据第二段,以下哪项正确?\\
\textbf{选项}: 
\begin{enumerate}[label=\Alph*)]
    \item 并非所有人都有美好未来的机会
    \item 绘画最能提升情绪
    \item 语言和音乐一样有用
    \item 看到人性光明面是碰运气
\end{enumerate}
\textbf{答案}: \textbf{C} \\
\textbf{依据}:
\begin{itemize}
    \item 原文明确表述:"Language is often useful for this, \textbf{and music}"
    \item 选项B错误原因:绘画效果需满足"if you have the right receptors"
\end{itemize}

\subsubsection*{Question 3}
\textbf{题目}: 作者认为:\\
\textbf{选项}: 
\begin{enumerate}[label=\Alph*)]
    \item 谎言越大,生理压力越强
    \item 生理变化必然暴露说谎者
    \item 抗压能力强的人说谎不易被发现
    \item 说谎压力程度取决于说谎目的
\end{enumerate}
\textbf{答案}: \textbf{D} \\
\textbf{依据}:
\begin{itemize}
    \item 文本定位:第五段"even when we do it for protection, or relief..."
    \item 逻辑推断:不同动机(purpose)对应不同心理负担(strain)
\end{itemize}

\subsubsection*{Question 4}
\textbf{题目}: 作者认为测谎仪提供的信息是:\\
\textbf{选项}: 
\begin{enumerate}[label=\Alph*)]
    \item 琐碎的
    \item 本质的
    \item 令人惊讶的
    \item 错误的
\end{enumerate}
\textbf{答案}: \textbf{A} \\
\textbf{依据}:
\begin{itemize}
    \item 作者评价:"a petty thing"(微不足道)
    \item 但强调其启示性价值:"tell us to look deeper"
\end{itemize}

\subsubsection*{Question 5}
\textbf{题目}: 作者传递的核心信息是:\\
\textbf{选项}: 
\begin{enumerate}[label=\Alph*)]
    \item 彼此诚实至关重要
    \item 说谎者损人利己
    \item 测谎仪可记录神经脉冲
    \item 先进测谎仪将遏制谎言
\end{enumerate}
\textbf{答案}: \textbf{A} \\
\textbf{依据}:
\begin{itemize}
    \item 核心论点:"biologically designed to be truthful"
    \item 结论句:"Trust is fundamental requirement for our existence"
\end{itemize}

\subsection*{Passage Two 答案解析}

\subsubsection*{Question 6}
\textbf{题目}: 患者被认为适合该实验的原因是 \\
\textbf{选项}: 
\begin{enumerate}[label=\Alph*)]
    \item 他自愿参与实验
    \item 他的感染在最早阶段被发现
    \item 他检测呈阳性并急于接受治疗
    \item 他对感染反应过度且过于谨慎
\end{enumerate}
\textbf{答案}: \textbf{B} \\
\textbf{依据}:
\begin{itemize}
    \item 关键句:"Because his patient acted on this suspicion, he was perfect for the study"
    \item 语境分析:患者因在急性期(acute phase)及时检测而被选为研究对象
\end{itemize}

\subsubsection*{Question 7}
\textbf{题目}: 根据文章,实验的初步发现是 \\
\textbf{选项}: 
\begin{enumerate}[label=\Alph*)]
    \item 及时治疗对患者控制感染至关重要
    \item 研究中八名患者的感染均得到控制
    \item 停止所有药物会更快破坏患者的免疫系统
    \item 免疫系统在早期治疗后无需药物即可发挥作用
\end{enumerate}
\textbf{答案}: \textbf{D} \\
\textbf{依据}:
\begin{itemize}
    \item 数据支撑:"five of the eight have now been off treatment... their infections are still well under control"
    \item 研究者结论:"the immune system can get the upper hand against the virus"
\end{itemize}

\subsubsection*{Question 8}
\textbf{题目}: 作者对实验初步发现的态度是 \\
\textbf{选项}: 
\begin{enumerate}[label=\Alph*)]
    \item 积极
    \item 消极
    \item 中立
    \item 不确定
\end{enumerate}
\textbf{答案}: \textbf{A} \\
\textbf{依据}:
\begin{itemize}
    \item 情感词汇:"tantalizing"(诱人的),"encouraging"(鼓舞人心的)
    \item 转折强调:"that concern wasn't borne out"(担忧未被证实)
\end{itemize}

\subsection*{Passage Three 答案解析}

\subsubsection*{Question 9}
\textbf{题目}: 文中提到的 "finicky eaters" 指哪类人? \\
\textbf{选项}: 
\begin{enumerate}[label=\Alph*)]
    \item 不规律饮食的人
    \item 讨厌吃蔬菜的人
    \item 对许多食物挑剔的人
    \item 喜欢奶酪饼干的人
\end{enumerate}
\textbf{答案}: \textbf{C} \\
\textbf{依据}:
\begin{itemize}
    \item 关键句:"they may help, particularly with finicky eaters"(第二段第二句)
    \item 上下文推断:"finicky" 对应 "挑食/挑剔",选项 C 最符合语境
\end{itemize}

\subsubsection*{Question 10}
\textbf{题目}: 本文的主要目的是 \\
\textbf{选项}: 
\begin{enumerate}[label=\Alph*)]
    \item 描述 EnfaGrow 的特点
    \item 提供儿童健康建议
    \item 推广 EnfaGrow 的销售
    \item 解释 EnfaGrow 的使用方法
\end{enumerate}
\textbf{答案}: \textbf{C} \\
\textbf{依据}:
\begin{itemize}
    \item 品牌定位:开篇即介绍 Mead Johnson 公司推出新产品 EnfaGrow(第二段)
    \item 宣传性语言:"Toddlers will definitely like..."(第三段首句)
    \item 平衡表述:虽提及营养建议,但最终落脚点在鼓励试用新产品(末段)
\end{itemize}

\subsection*{Passage Four 答案解析}

\subsubsection*{Question 11}
\textbf{题目}: Cooper 与 Warburton 先生的关系是? \\
\textbf{选项}: 
\begin{enumerate}[label=\Alph*)]
    \item 客人与主人
    \item 客人与酒店经理
    \item 客人与男仆
    \item 客人与侍者
\end{enumerate}
\textbf{答案}: \textbf{A} \\
\textbf{依据}:
\begin{itemize}
    \item 关键场景:"Mr. Warburton noticed... he went into the dining room to see that the table was properly laid"(主人布置餐桌)
    \item 互动特征:"Hulloa, you're all dressed up," said Cooper(客人对主人正式着装的评论)
\end{itemize}

\subsubsection*{Question 12}
\textbf{题目}: 从文中可推断 Warburton 先生具有的特点是? \\
\textbf{选项}: 
\begin{enumerate}[label=\Alph*)]
    \item 偏爱休闲服饰
    \item 行事粗心
    \item 脾气暴躁
    \item 注重礼节
\end{enumerate}
\textbf{答案}: \textbf{D} \\
\textbf{依据}:
\begin{itemize}
    \item 细节描写:"dressed as formally as though he were dining at his club... always dress for dinner"(坚持正装晚餐)
    \item 对比强调:"Especially when I'm alone"(独处时仍保持仪式感)
\end{itemize}

\subsubsection*{Question 13}
\textbf{题目}: Cooper 与 Warburton 先生的主要差异体现在? \\
\textbf{隐含选项}:
\begin{itemize}
    \item 着装风格:Cooper 的邋遢("ragged jacket")vs. Warburton 的精致("white dinner-jacket")
    \item 行为模式:Cooper 的随意("very nearly put on a sarong")vs. Warburton 的刻板("frigid stare")
\end{itemize}
\textbf{核心冲突}: \textbf{对待社交礼仪的态度差异}


\subsection*{Passage Five 答案解析}

\subsubsection*{Question 14}
\textbf{题目}: 作者在文末暗示了什么? \\
\textbf{选项}: 
\begin{enumerate}[label=\Alph*)]
    \item 已找到辐射问题的解决方案
    \item 解决辐射问题是不可能的
    \item 多数方案不切实际
    \item 当前方案部分有效
\end{enumerate}
\textbf{答案}: \textbf{C} \\
\textbf{依据}:
\begin{itemize}
    \item 结论句:"there is at present no complete solution"(第四段末句)
    \item 方案分析:阿波罗飞船的屏蔽方案仅适用于短途("not possible for longer journeys"),磁场方案目前不可行("not in fact possible at present")
\end{itemize}

\subsubsection*{Question 15}
\textbf{题目}: 吸收辐射的化学层名称是? \\
\textbf{选项}: 
\begin{enumerate}[label=\Alph*)]
    \item 范艾伦带
    \item 大气层
    \item 臭氧层
    \item 太阳辐射
\end{enumerate}
\textbf{答案}: \textbf{C} \\
\textbf{依据}:
\begin{itemize}
    \item 定义句:"the ozonosphere... a belt of the chemical ozone between 12 and 21 miles from the ground which absorbs all the radiation"(第二段末句)
    \item 功能对比:Van Allen Belts 属于辐射源(第二段第三句),臭氧层属于防护层
\end{itemize}

\subsubsection*{Question 16}
\textbf{题目}: 辐射短期与长期效应的主要区别是? \\
\textbf{隐含选项}:
\begin{itemize}
    \item 短期效应:"merely unpleasant"(第三段第二句)
    \item 长期效应:"extremely serious, even leading to death"(第三段末句)
\end{itemize}
\textbf{核心区别}: \textbf{健康后果的严重程度}


\subsection*{Passage Six 答案解析}

\subsubsection*{Question 17}
\textbf{题目}: "巨型履带牵引车"的功能是? \\
\textbf{选项}: 
\begin{enumerate}[label=\Alph*)]
    \item 帮助航天飞机起飞
    \item 将航天飞机运至发射台
    \item 将航天飞机运至肯尼迪航天中心
    \item 发射航天飞机进入太空
\end{enumerate}
\textbf{答案}: \textbf{B} \\
\textbf{依据}:
\begin{itemize}
    \item 关键描述:"rolling to its lift-off site on a gigantic crawler-tractor... creep the 5.5 kilometers from the Kennedy Space Center"(第二段第二句)
    \item 逻辑关系:履带车负责运输航天器至发射位(lift-off site),而非直接参与发射
\end{itemize}

\subsubsection*{Question 18}
\textbf{题目}: 哥伦比亚号的太空用途不包括? \\
\textbf{选项}: 
\begin{enumerate}[label=\Alph*)]
    \item 进行太空调查
    \item 寻找矿藏
    \item 研究天气
    \item 协助轨道望远镜运行
\end{enumerate}
\textbf{答案}: \textbf{A} \\
\textbf{依据}:
\begin{itemize}
    \item 原文列举用途:"surveys of the Earth and oceans"(第一段第三句)——明确限定为地球与海洋
    \item 排除法:选项B/C/D均在第一段中被明确提及(mineral deposits/weather/telescope)
\end{itemize}

\subsubsection*{Question 19}
\textbf{题目}: 哥伦比亚号对太空探索的重要性体现在? \\
\textbf{隐含答案}:
\begin{itemize}
    \item 革命性设计:"unlike ordinary rockets... could now rocket into space and then fly back"(第三段)
    \item 技术突破:"first lift-off in April 1981... perfect landing" 实现可重复使用航天器
\end{itemize}
\textbf{核心贡献}: \textbf{开创可重复使用航天器的新纪元}



\subsection*{Passage Seven 答案解析}

\subsubsection*{YES/NO/NOT GIVEN 判断}
\textbf{判断标准}:
\begin{itemize}
    \item \textbf{YES}: 陈述与作者观点一致
    \item \textbf{NO}: 陈述与作者观点矛盾
    \item \textbf{NOT GIVEN}: 无法从文中推断
\end{itemize}

\subsubsection*{Question 20}
\textbf{陈述}: 媒体不赞成使用安全气囊 \\
\textbf{答案}: \textbf{NO} \\
\textbf{依据}:
\begin{itemize}
    \item 原文描述:"problems... have received considerable play in the press"(第二段第三句)——仅说明媒体广泛报道问题,未表达反对立场
\end{itemize}

\subsubsection*{Question 21}
\textbf{陈述}: 低速行驶(<16mph)乘客不易受伤 \\
\textbf{答案}: \textbf{NO} \\
\textbf{依据}:
\begin{itemize}
    \item 数据对比:"many fatalities occurred at speeds less than 16 mph"(第三段第二句)
    \item 作者观点:低速事故中未系安全带者仍存在高风险
\end{itemize}

\subsubsection*{Question 22}
\textbf{陈述}: 安全气囊致死均由追尾碰撞引起 \\
\textbf{答案}: \textbf{NO} \\
\textbf{依据}:
\begin{itemize}
    \item 死亡原因分析:主要与乘员位置/安全带使用相关(第三段)
    \item 追尾碰撞研究单独讨论:涉及挥鞭伤(whiplash)而非气囊致死(第五段)
\end{itemize}

\subsubsection*{Question 23}
\textbf{陈述}: 安全气囊伤害年损失\$100亿 \\
\textbf{答案}: \textbf{NO} \\
\textbf{依据}:
\begin{itemize}
    \item \$100亿数据对应:"neck strain/sprain... \$10 billion annually"(第五段第一句)
    \item 明确区分:颈部损伤 ≠ 安全气囊直接致伤
\end{itemize}

\subsubsection*{Question 24}
\textbf{陈述}: 历史安全设计侧重成年男性 \\
\textbf{答案}: \textbf{YES} \\
\textbf{依据}:
\begin{itemize}
    \item 原文引用:"Historically emphasized the average adult male"(第六段第一句)
\end{itemize}

\subsubsection*{Question 25}
\textbf{陈述}: 儿童车祸机理从未被系统研究 \\
\textbf{答案}: \textbf{YES} \\
\textbf{依据}:
\begin{itemize}
    \item 研究性质描述:"first comprehensive investigation"(第六段第二句)
    \item 时间标志词:"Looking ahead" 与 "Historically" 形成对比
\end{itemize}



\subsection*{词汇与结构答案解析}

\subsubsection*{26-35题}
\begin{enumerate}[label=\textbf{\arabic*.}]
    \item \textbf{D) flexible} 
    \begin{itemize}
        \item 语境对比:"set in her ways"(固执)vs "flexible"(灵活)
        \item 固定搭配:have a flexible attitude
    \end{itemize}
    
    \item \textbf{C) treated} 
    \begin{itemize}
        \item 医疗场景:"treated for..." 治疗具体病症
        \item 排除干扰项:cure(治愈疾病)/heal(伤口愈合)
    \end{itemize}
    
    \item \textbf{B) vacated} 
    \begin{itemize}
        \item 酒店用语:vacate a room(退房)
        \item 词义辨析:evacuate(紧急疏散)/desert(遗弃)
    \end{itemize}
    
    \item \textbf{D) pull through} 
    \begin{itemize}
        \item 医疗短语:pull through = 恢复健康
        \item 排除项:pull up(停车)/pull in(进站)
    \end{itemize}
    
    \item \textbf{A) throw off} 
    \begin{itemize}
        \item 固定搭配:throw off a cough(摆脱咳嗽)
        \item 动作持续性:与 "long time" 呼应
    \end{itemize}
    
    \item \textbf{A) expired} 
    \begin{itemize}
        \item 证件术语:passport expires(护照过期)
        \item 时间线索:"last month" 对应过期状态
    \end{itemize}
    
    \item \textbf{B) acquaint} 
    \begin{itemize}
        \item 固定搭配:acquaint sb with sth(使熟悉)
        \item 语义需求:调查目的为信息传递
    \end{itemize}
    
    \item \textbf{B) eligible} 
    \begin{itemize}
        \item 法律用语:eligible to join(具备资格)
        \item 年龄限制:under 21 对应资格问题
    \end{itemize}
    
    \item \textbf{C) restored} 
    \begin{itemize}
        \item 文物修复:restore to original splendor(恢复原貌)
        \item 区分:renovate(翻新)/repair(修理)
    \end{itemize}
    
    \item \textbf{A) distracts} 
    \begin{itemize}
        \item 固定搭配:distract from work(分散注意力)
        \item 程度递进:与 "cannot bear" 形成因果关系
    \end{itemize}
\end{enumerate}

\subsubsection*{36-48题}
\begin{enumerate}[resume*]
    \item \textbf{D) get by} 
    \begin{itemize}
        \item 生存需求:get by without...(没有...勉强过活)
        \item 生活场景:与香烟数量关联
    \end{itemize}
    
    \item \textbf{A) standing} 
    \begin{itemize}
        \item 选举术语:stand as candidate(参选)
        \item 介词搭配:as + 职位
    \end{itemize}
    
    \item \textbf{C) tried} 
    \begin{itemize}
        \item 法律程序:tried in court(庭审)
        \item 案件类型:minor cases 对应庭审流程
    \end{itemize}
    
    \item \textbf{C) minimum} 
    \begin{itemize}
        \item 法律定义:minimum voting age(最低投票年龄)
        \item 数字限定:18岁为下限
    \end{itemize}
    
    \item \textbf{D) juvenile} 
    \begin{itemize}
        \item 司法术语:juvenile courts(少年法庭)
        \item 年龄限定:under 18 对应少年司法
    \end{itemize}
    
    \item \textbf{C) controversy} 
    \begin{itemize}
        \item 语义转折:in spite of controversy(尽管有争议)
        \item 事件性质:与 "show up" 形成对比
    \end{itemize}
    
    \item \textbf{D) eccentrics} 
    \begin{itemize}
        \item 城市特征:eccentrics(怪人)对应 odd information
        \item 文化意象:纽约城市形象
    \end{itemize}
    
    \item \textbf{A) certify} 
    \begin{itemize}
        \item 学术认证:certify ability(证明能力)
        \item 成绩功能:与 performance 形成对比
    \end{itemize}
    
    \item \textbf{A) ascertain} 
    \begin{itemize}
        \item 调查动作:ascertain truth(查明真相)
        \item 警方职责:重建事实
    \end{itemize}
    
    \item \textbf{B) compensation} 
    \begin{itemize}
        \item 法律赔偿:seek compensation(索赔)
        \item 事故后果:空难家属诉求
    \end{itemize}
    
    \item \textbf{B) setbacks} 
    \begin{itemize}
        \item 项目进展:have setbacks(遭遇挫折)
        \item 机械故障:与初期顺利对比
    \end{itemize}
    
    \item \textbf{D) mourn} 
    \begin{itemize}
        \item 情感表达:mourn for deceased(哀悼逝者)
        \item 时间跨度:30年持续哀伤
    \end{itemize}
    
    \item \textbf{B) staple} 
    \begin{itemize}
        \item 经济作物:staple crop(主要作物)
        \item 收入来源:bring money 对应经济地位
    \end{itemize}
\end{enumerate}

\subsubsection*{49-55题}
\begin{enumerate}[resume*]
    \item \textbf{A) discard} 
    \begin{itemize}
        \item 搬家场景:discard furniture(丢弃家具)
        \item 空间需求:more room 对应减负
    \end{itemize}
    
    \item \textbf{C) bald} 
    \begin{itemize}
        \item 生理特征:bald patch(秃斑)
        \item 掩饰动作:sweep hair over
    \end{itemize}
    
    \item \textbf{C) overwhelmed} 
    \begin{itemize}
        \item 情感表达:feel overwhelmed(不堪重负)
        \item 程度副词:how 强调强烈感受
    \end{itemize}
    
    \item \textbf{B) deemed} 
    \begin{itemize}
        \item 法律效力:be deemed to have done(视为自动...)
        \item 截止期限:by end of month 的后果
    \end{itemize}
    
    \item \textbf{D) tempted} 
    \begin{itemize}
        \item 心理矛盾:tempted by food(受美食诱惑)
        \item 让步状语:although on diet 形成冲突
    \end{itemize}
    
    \item \textbf{C) jeopardize} 
    \begin{itemize}
        \item 后果预测:jeopardize future(危及生存)
        \item 政策影响:三方面连锁反应
    \end{itemize}
    
    \item \textbf{D) retrieve} 
    \begin{itemize}
        \item 计算机功能:store and retrieve(存储检索)
        \item 信息管理:efficiently 强调效率
    \end{itemize}
\end{enumerate}

%
\subsection*{完形填空答案解析}

\subsubsection*{56-65题}
\begin{enumerate}[label=\textbf{\arabic*.}]
    \item \textbf{B) browse} 
    \begin{itemize}
        \item 搭配语境:"casually \_\_ through credit card purchases"(随意浏览信用卡消费)
        \item 词义辨析:browse(浏览)vs glance(瞥见)
    \end{itemize}
    
    \item \textbf{C) preferences} 
    \begin{itemize}
        \item 购物特征:shopping preferences(消费偏好)
        \item 排除干扰项:aptitudes(天赋)为能力倾向
    \end{itemize}
    
    \item \textbf{B) watch} 
    \begin{itemize}
        \item 监视行为:watch without permission(未经允许监视)
        \item 语义强度:watch > see(普通看)
    \end{itemize}
    
    \item \textbf{B) intended} 
    \begin{itemize}
        \item 意愿表达:"never intended to be seen"(从未打算被看到)
        \item 语法结构:被动式搭配不定式
    \end{itemize}
    
    \item \textbf{B) red-handed} 
    \begin{itemize}
        \item 固定搭配:caught red-handed(当场抓获)
        \item 数字时代隐喻:类比传统犯罪现场
    \end{itemize}
    
    \item \textbf{D) boundaries} 
    \begin{itemize}
        \item 心理学概念:healthy boundaries(健康界限)
        \item 语境呼应:"reveal yourself... at appropriate times"
    \end{itemize}
    
    \item \textbf{A) in} 
    \begin{itemize}
        \item 阶段表达:reveal in stages(分阶段透露)
        \item 介词搭配:in + 抽象阶段
    \end{itemize}
    
    \item \textbf{B) digital} 
    \begin{itemize}
        \item 数字痕迹:"digital bread crumbs"(电子足迹)
        \item 时代特征:与后文Google搜索呼应
    \end{itemize}
    
    \item \textbf{C) strangers} 
    \begin{itemize}
        \item 前文复现:首段首句 "someday a stranger"
        \item 威胁来源:陌生人重构个人信息
    \end{itemize}
    
    \item \textbf{B) search} 
    \begin{itemize}
        \item 搜索引擎:Google search(谷歌搜索)
        \item 功能对应:reveal information(暴露信息)
    \end{itemize}
\end{enumerate}

\subsubsection*{66-75题}
\begin{enumerate}[resume*]
    \item \textbf{A) Like it or not} 
    \begin{itemize}
        \item 固定表达:Like it or not(无论喜欢与否)
        \item 逻辑连接:引出客观事实
    \end{itemize}
    
    \item \textbf{D) matter} 
    \begin{itemize}
        \item 核心问题:"Does that matter?"(这重要吗?)
        \item 篇章主旨:隐私是否值得关注
    \end{itemize}
    
    \item \textbf{B) answer} 
    \begin{itemize}
        \item 问答结构:前句 "key question" 对应 answer
        \item 调查结论:多数人答案是否定
    \end{itemize}
    
    \item \textbf{A) losing} 
    \begin{itemize}
        \item 动态过程:losing privacy(隐私流失)
        \item 调查反馈:60\%感觉隐私流失
    \end{itemize}
    
    \item \textbf{B) pessimism} 
    \begin{itemize}
        \item 情感倾向:overwhelming pessimism(普遍悲观)
        \item 数据支持:60\%受访者负面感受
    \end{itemize}
    
    \item \textbf{C) slipping} 
    \begin{itemize}
        \item 渐进消失:privacy is slipping away(隐私逐渐消失)
        \item 比喻使用:slipping 强调不可控性
    \end{itemize}
    
    \item \textbf{A) another} 
    \begin{itemize}
        \item 对比结构:say one thing and do another(言行不一)
        \item 固定搭配:one...another...
    \end{itemize}
    
    \item \textbf{D) behaviors} 
    \begin{itemize}
        \item 行为改变:change behaviors(改变行为)
        \item 现实矛盾:与隐私担忧形成反差
    \end{itemize}
    
    \item \textbf{C) down} 
    \begin{itemize}
        \item 短语动词:turn down a discount(拒绝优惠)
        \item 代价选择:为隐私放弃经济利益
    \end{itemize}
    
    \item \textbf{C) track} 
    \begin{itemize}
        \item 技术追踪:track movements(追踪行踪)
        \item 系统功能:EZ-Pass记录车辆轨迹
    \end{itemize}
\end{enumerate}

\newpage
\section*{2021}

\subsection*{阅读理解解析(Passage One)}

\subsubsection*{问题1}
\textbf{题目:} 在文章第1段最后一句中,"our particular creature"指代的是:\\
A. 对某物的恐惧 \\
B. 公众嘲笑 \\
C. 凶猛野兽 \\
D. 身体疼痛

\textbf{答案:} \boxed{A} \\
\textbf{解析:} 原文首段列举不同恐惧类型(怕黑、怕疼等),最后用"particular creature"隐喻\underline{个性化的恐惧对象}。通过"for all of us our particular creature waits..."可知,此处用"creature"指代不同人特有的恐惧类型。

\subsubsection*{问题2}
\textbf{题目:} 恐惧是一种有用的情绪,因为它可以:\\
A. 加快心跳和反应速度 \\
B. 帮助快速应对危险并自我保护 \\
C. 向血液中注入大量肾上腺素 \\
D. 刺激体内许多生理变化

\textbf{答案:} \boxed{B} \\
\textbf{解析:} 第二段通过"fuel life-saving flight"明确功能指向\underline{生存保护},而A/C/D选项均为生理现象描述,属于B选项的\underline{实现手段}而非本质原因。

\subsubsection*{问题3}
\textbf{题目:} 恐惧何时会成为问题?\\
A. 认为危险比实际更大时 \\
B. 无法承受危险时 \\
C. 危险是心理而非生理时 \\
D. 未做好充分准备时

\textbf{答案:} \boxed{A} \\
\textbf{解析:} 原文第二段末句"disproportional to the danger"直接对应A选项的\underline{风险认知偏差}。注意C选项为正常恐惧作用场景(见第二段"psychological rather than physical"部分)。

\subsubsection*{问题4}
\textbf{题目:} 新生儿对门巨响的不同反应表明:\\
A. 人们有时对噪音充耳不闻 \\
B. 人对刺激的反应非遗传特征 \\
C. 某些人对噪音非常敏感 \\
D. 某些人天生更容易受危险影响

\textbf{答案:} \boxed{D} \\
\textbf{解析:} 第三段"inherited a tendency to be more sensitive"明确说明\underline{先天遗传敏感性}的存在,D选项中的"inherently"与此直接对应。

\subsubsection*{问题5}
\textbf{题目:} 心理学家发现,我们后期的恐惧主要由\_\_\_\_\_\_决定:\\
A. 父母生活方式 \\
B. 学校教育 \\
C. 早期经历 \\
D. 家庭教育

\textbf{答案:} \boxed{C} \\
\textbf{解析:} 第四段首句"early experiences and relationships strongly shape and determine our later fears"中,\underline{early experiences}即对应选项C。注意D选项"家庭教育"属于C的子集,而原文包含更广泛的关系影响。


\subsection*{阅读理解解析(Passage Two)}

\subsubsection*{问题6}
\textbf{题目:} 根据文章,\\
A. 人们为让孩子获得理想教育支付高额费用 \\
B. 私立学校爆满,家长需额外付费才能入学 \\
C. 所有家长都拒绝送孩子去公立学校 \\
D. 私立学校如果是走读制则有长等候名单

\textbf{答案:} \boxed{A} \\
\textbf{解析:} 首段首句"parents seem prepared to take on the formidable extra cost"直接说明家长愿意承担高额费用,对应A选项。B选项"额外付费入学"在原文未提及(仅提到等候名单长)。

\subsubsection*{问题7}
\textbf{题目:} 为同时受益于公立与私立教育系统,许多孩子:\\
A. 在付费学校就读一段时间后进入公立学校 \\
B. 主要在公立系统接受教育 \\
C. 在公立与私立学校间交替就读 \\
D. 进入教育模式正改变的走读学校

\textbf{答案:} \boxed{A} \\
\textbf{解析:} 第二段明确指出"spending primary years in State system, then moving into fee-paying school, and sometimes finishing off in a State school",即最终可能返回公立学校,对应A选项。D选项的"教育模式改变"是走读学校现状,非问题核心。

\subsubsection*{问题8}
\textbf{题目:} 根据ISIS年度调查,\\
A. 私立教育高成本正在阻止许多家长 \\
B. 私立学校费用年增长率仅约3\% \\
C. 私立学校费用增长勉强跟上通胀 \\
D. 私立学校不支付更高薪资和行政成本

\textbf{答案:} \boxed{C} \\
\textbf{解析:} 末段明确说明"increases of less than 3\% a term... keep up with inflation",但"does not take care of higher salaries and soaring costs",故C正确。注意B选项的"年增长率"错误(原文为每学期)。

\subsection*{阅读理解解析(Passage Three)}

\subsubsection*{问题9}
\textbf{题目:} 轻微中风引起医生特别关注的原因是:\\
A. 可有效治愈 \\
B. 无明显症状 \\
C. 此前研究较少 \\
D. 难以治疗

\textbf{答案:} \boxed{B} \\
\textbf{解析:} 原文第三段明确指出TIAs具有\underline{隐蔽性损伤(stealth of their damage)}且\underline{及时治疗极有效},对应B选项"无明显症状"。注意D选项与文中"dramatic effectiveness of timely treatment"矛盾。

\subsubsection*{问题10}
\textbf{题目:} 下列哪句话最能总结文章主旨?\\
A. 呼吁关注轻微中风的后果 \\
B. 全面描述轻微中风 \\
C. 解释轻微中风发生的时机和方式 \\
D. 讨论治疗中风的更好方法

\textbf{答案:} \boxed{A} \\
\textbf{解析:} 文章核心围绕\underline{ministrokes的危害性}(如"precursors of major stroke")和\underline{公众意识不足}("underdiagnosed"),并通过数据强调\underline{及时诊断的重要性},符合A选项的"关注后果"定位。D选项范围过大(文章侧重ministrokes而非所有中风)。

%
\subsection*{阅读理解解析(Passage Four)}

\subsubsection*{问题11}
\textbf{题目:} 为何打字员阅读副本时能准确打字?\\
A. 因集中注意力于打字 \\
B. 因打字是自动化工作 \\
C. 因训练帮助建立脑内模式 \\
D. 因手指自动敲击正确键

\textbf{答案:} \boxed{C} \\
\textbf{解析:} 第二段明确指出"requires intense concentration and repetition during the learning phase to establish a pattern in the brain",强调\underline{训练建立脑内模式}是核心机制。D选项描述的是结果现象,非根本原因。

\subsubsection*{问题12}
\textbf{题目:} 打字员疲劳时会发生什么?\\
A. 常失去自动化能力 \\
B. 无法思考当前行为 \\
C. 只能自动打字 \\
D. 需专注工作故无错

\textbf{答案:} \boxed{A} \\
\textbf{解析:} 第三段"beyond a certain point the automaticity is lost"直接对应A选项。B选项"无法思考"与原文"thinking about hitting the keys"矛盾。

\subsubsection*{问题13}
\textbf{题目:} 根据文章,推理能力的关键因素是什么?(简答题)\\
\textbf{答案:} \boxed{\text{Arousal level (警觉水平)}} \\
\textbf{解析:} 末段研究显示当arousal level提升时,"increase the accuracy of looking for rare events"且医生"became better at tests of grammatical reasoning",表明警觉水平是推理能力的关键调节因素。

\subsection*{阅读理解解析(Passage Five)}

\subsubsection*{问题14}
\textbf{题目:} 第二段的主旨是:\\
A. 我们有很多证据显示心理如何运作 \\
B. 普遍认为身心会同时影响彼此 \\
C. 我们的身体受思想和情感影响 \\
D. 我们对身心关系了解不足

\textbf{答案:} \boxed{B} \\
\textbf{解析:} 第二段明确指出"body and mind are constantly interacting in some way is now accepted by most psychologists",直接对应B选项的"普遍接受"表述。C选项仅描述单向影响,未涵盖"every change within the nervous system should have some psychological effect"的双向交互。

\subsubsection*{问题15}
\textbf{题目:} 第四段中,GSR被描述为:\\
A. 测谎仪的别名 \\
B. 测量体内电量 \\
C. 仅测量强烈情绪 \\
D. 测量通过皮肤的电流量

\textbf{答案:} \boxed{D} \\
\textbf{解析:} 原文明确说明GSR通过"pass a weak current between the electrodes"并记录"changes in the current",故D选项正确。A选项错误(GSR是测谎仪使用的技术而非别名),B选项混淆了电流变化与"电量"概念。

\subsubsection*{问题16}
\textbf{题目:} 文章传达的主要信息是什么?(简答题)\\
\textbf{答案:} \boxed{\text{Mind-body interaction (身心交互作用)}} \\
\textbf{解析:} 全文围绕\underline{身心双向影响}展开,通过情绪状态(如恐惧、愤怒)的生理反应和GSR测量技术佐证这一核心论点。第二段"body and mind are constantly interacting"为直接主旨句。

\subsection*{阅读理解解析(Passage Six)}

\subsubsection*{问题17}
\textbf{题目:} 判断儿童水平时,测试分数需与谁的分数比较?\\
A. 同校许多儿童 \\
B. 同一儿童不同年龄 \\
C. 更大和更小儿童 \\
D. 同龄其他儿童

\textbf{答案:} \boxed{D} \\
\textbf{解析:} 首段明确要求"compared with the average achieved by boys of thirteen in that test",直接对应\underline{同龄人标准}。C选项混淆了年龄跨度,原文强调同年龄组比较。

\subsubsection*{问题18}
\textbf{题目:} 被称为"三年滞后"的9岁男孩,其心理年龄是?\\
A. 3岁 \quad B. 6岁 \quad C. 9岁 \quad D. 12岁

\textbf{答案:} \boxed{B} \\
\textbf{解析:} 根据第三段定义,"three years retarded"表示\[
\text{心理年龄} = \text{实际年龄} - \text{滞后年数} = 9 - 3 = 6
\]。需注意题干中实际年龄为9岁(非原文的12岁案例)。

\subsubsection*{问题19}
\textbf{题目:} IQ如何计算?(简答题)\\
\textbf{答案:} \boxed{\text{(Mental Age / Chronological Age) \times 100}} \\
\textbf{解析:} 末段明确指出"The 'I.Q.' is the mental ratio multiplied by 100",其中mental ratio即为\[
\text{mental ratio} = \frac{\text{mental age}}{\text{chronological age}}
\]。例如:9岁心理年龄/12岁实际年龄=0.75,IQ=75。

\subsection*{判断解析(Passage Seven)}

\subsubsection*{问题20}
\textbf{陈述:} 存在科学证据证明手机会引起加油站爆炸 \\
\textbf{答案:} \boxed{NO} \\
\textbf{解析:} 首段明确说明"the concern rests not on scientific evidence of any danger",直接否定科学证据的存在。Burgess的结论基于社会学分析而非物理证据。

\subsubsection*{问题21}
\textbf{陈述:} Piper Alpha灾难由手机火花引起 \\
\textbf{答案:} \boxed{NO} \\
\textbf{解析:} 第二段明确指出Piper Alpha是1988年\underline{石油平台爆炸事故},与手机使用无关。安全运动是该事件的响应措施,但原文未提及手机是事故原因。

\subsubsection*{问题22}
\textbf{陈述:} 石油公司使用扶手是为防止爆炸 \\
\textbf{答案:} \boxed{NO} \\
\textbf{解析:} 第二段描述"use handrails"属于\underline{办公场所安全措施}(防止摔伤),与防爆措施无逻辑关联。题干存在因果关系错误。

\subsubsection*{问题23}
\textbf{陈述:} 90年代末公众相信手机不会引起爆炸 \\
\textbf{答案:} \boxed{NO} \\
\textbf{解析:} 第三段指出"it was too late. The myth had taken hold",说明虽然厂商澄清,但\underline{公众认知已固化}。后续段落继续描述禁令持续的现象,佐证公众未改变观念。

\subsubsection*{问题24}
\textbf{陈述:} 加油站火灾增多因电气设备普及 \\
\textbf{答案:} \boxed{NO} \\
\textbf{解析:} 第四段Richard Coates的研究结论指出"sparks were caused by static electricity, not electrical equipment"。火灾增加与\underline{静电}相关,与电气设备使用无关。

\subsubsection*{问题25}
\textbf{陈述:} 互联网是手机引发爆炸观念盛行的原因 \\
\textbf{答案:} \boxed{YES} \\
\textbf{解析:} 第五段详细描述"hoax memos on the Internet...mistakenly attribute them to mobile phones",明确互联网传播错误信息的作用。题干因果关系与原文一致。

\subsection*{第II节 词汇与结构(26-55题)}
\begin{enumerate}
    \item[26.] \textbf{C. warn} \\ 
    解析:\textit{alert}(警报)与\textit{warn}(警告)构成同义替换,均表示预警功能
    
    \item[27.] \textbf{D. rushing} \\ 
    解析:\textit{scurrying}描述蚂蚁快速移动,与\textit{rushing}(急速行动)动作特征匹配
    
    \item[28.] \textbf{C. different} \\ 
    解析:\textit{diverse}(多样化)在"six diverse cultures"中需理解为不同文化群体
    
    \item[29.] \textbf{C. group} \\ 
    解析:\textit{cluster}指生物群体聚集现象,\textit{group}为直接对应词
    
    \item[30.] \textbf{C. available} \\ 
    解析:黏土资源是否"accessible"(可获取)对应资源可用性(\textit{available})
    
    \item[31.] \textbf{B. explain} \\ 
    解析:\textit{account for}固定短语表"解释",与科学说明需求直接关联
    
    \item[32.] \textbf{C. exposed} \\ 
    解析:\textit{subjected}在此处构成"be subjected to"结构,与"exposed to"(暴露于)同义
    
    \item[33.] \textbf{D. observed} \\ 
    解析:研究者通过\textit{noted}(记录)行为完成观察,与\textit{observed}(观察)形成动作链
    
    \item[34.] \textbf{A. basic nature} \\ 
    解析:\textit{essence}(本质)在句中指民间音乐的核心属性,对应基础特性表述
\end{enumerate}

\subsection*{第III节 完形填空(56-75题)}
\begin{tabular}{|c|c|l|}
\hline
题号 & 答案 & 解析 \\
\hline
56 & C & 首段主题词复现(后文多次出现\textit{gifted children}) \\
57 & A & \textit{have little good to say}固定否定结构 \\
58 & B & "achieved distinction"表取得卓越成就 \\
59 & A & MacArthur奖获得者语义要求 \\
60 & B & 创造性成就(\textit{creative})与奖项定位匹配 \\
61 & D & \textit{collegiate}(大学阶段)与\textit{advanced programs}关联 \\
62 & B & 虚拟语气过去完成时(\textit{if}条件句) \\
63 & B & 倒装结构\textit{So did Churchill}承接前文否定 \\
64 & A & 史实匹配(丘吉尔就读英国精英学校) \\
65 & A & 教师视角负面评价链(后文\textit{arrogance}呼应) \\
66 & C & 学术体系(\textit{scholastic})认可度缺失 \\
67 & C & \textit{account for}(解释原因)固定搭配 \\
68 & D & \textit{not...but...}转折结构 \\
69 & A & 因果关系(\textit{unchallenging}导致失去兴趣) \\
70 & A & 原词复现(首段\textit{lack of fit}) \\
71 & B & 后文\textit{Nonconformity}直接对应 \\
72 & A & 师生冲突(\textit{lead to conflict}) \\
73 & B & 通用表达\textit{in any area}(任何领域) \\
74 & B & 程度副词\textit{far}修饰比较级 \\
75 & C & 比较结构\textit{more...than...} \\
\hline
\end{tabular}

\subsubsection*{关键语法现象}
\begin{itemize}
    \item 虚拟语气:题62使用\textit{had not been placed}表示与过去事实相反
    \item 倒装结构:题63的\textit{So did Churchill}承接前文否定陈述
    \item 比较级修饰:题74的\textit{far more likely}符合副词修饰规则
\end{itemize}

\newpage
\section*{2022}

\subsection*{Passage One: MAKING THE MOST OF TRENDS}
\begin{enumerate}
    \item[1.] \textbf{D) are unaware of the significant impact that trends have on consumers' lives} \\ 
    解析:首段第二句"managers often fail to recognize the less obvious but profound ways these trends are influencing consumers'...",\textit{profound}对应D选项中"significant impact"
    
    \item[2.] \textbf{C) safeguard its reputation as a manufacturer of luxury goods} \\ 
    解析:第三段"the most obvious reaction...would have risked cheapening the brand's image"说明Coach避免降价以保护品牌形象
    
    \item[3.] \textbf{A) It did not require Tesco to modify its core business activities} \\ 
    解析:第四段"Tesco has not abandoned its traditional retail offerings but augmented its business...",\textit{augmented}说明核心业务未改变
    
    \item[4.] \textbf{C) It was the type of strategy that would not have been possible in the past} \\ 
    解析:第五段"combine aspects of the product's existing value proposition with attributes addressing changes...land the company in an entirely new market space",数字技术整合是新时代特有策略
    
    \item[5.] \textbf{D) It was a handheld game that addressed people's concerns about unhealthy lifestyles} \\ 
    解析:第六段"counteracting the negatives, such as associations with lack of exercise and obesity",ME2通过计步器功能解决健康担忧
\end{enumerate}

\subsubsection*{核心论证结构}
\begin{tabular}{|l|l|}
\hline
\textbf{策略类型} & \textbf{对应案例} \\
\hline
Infuse and Augment & Coach推出Poppy系列/Tesco环保积分计划 \\
Combine and Transcend & Nike+运动套件 \\
Counteract and Reaffirm & iToys ME2游戏机 \\
\hline
\end{tabular}

\subsubsection*{高频命题点}
\begin{itemize}
    \item \textbf{例证关系}:问题2/3/4/5均需匹配案例与策略特征(如ME2对应健康问题)
    \item \textbf{否定强调}:问题1的"fail to recognize"与选项D的"unaware"形成同义转换
    \item \textbf{程度限定}:问题4中"entirely new market space"对应选项C的"not possible in the past"
\end{itemize}

\subsection*{Passage Two: Aviaphobia 飞行恐惧症研究}
\begin{enumerate}
    \item[6.] \textbf{B) even many frequent travelers are afraid of flying} \\ 
    解析:首段"paradoxically, even business people...necessitate frequent air travel"揭示讽刺点:需频繁飞行人群同样受恐惧困扰
    
    \item[7.] \textbf{A) ambitious} \\ 
    解析:第四段"overachievers and perfectionists...responsible and respectable positions"暗示高成就者的野心导致隐瞒恐惧
    
    \item[8.] \textbf{D) nightmares} \\ 
    解析:第三段心理学家列举的恐惧包含:heights(A)、dying(B)、claustrophobia(closed-in places),而选项C(losing control)虽在第四段提及,但属于独立现象而非心理学者列出的核心恐惧
\end{enumerate}

\subsubsection*{语义逻辑解密}
\begin{tabular}{|l|l|}
\hline
\textbf{关键词} & \textbf{对应考点} \\
\hline
ironic (题6) & 矛盾修辞:常飞人群反而患病 \\
closet white-knuckle flier (题7) & 隐喻手法暗示秘密性 \\
vicarious factor (题8) & 排除法需结合上下文隐含信息 \\
\hline
\end{tabular}

\subsubsection*{高频干扰项分析}
\begin{itemize}
    \item 题6选项C:"航空公司员工患病"虽为事实,但需注意题干要求"ironic"的核心矛盾点在于"常旅客vs恐惧"而非职业属性
    \item 题8选项C:"失去控制"虽在第四段出现,但属于独立现象解释(患者心理特质),非心理学家列出的直接恐惧来源
\end{itemize}

\subsection*{Passage Three: 珊瑚礁白化危机}
\begin{enumerate}
    \item[9.] \textbf{C) causes the corals to die} \\ 
    解析:第三段"heat boosts the plants' metabolism...corals will eventually die"明确指出水温升高导致珊瑚排出藻类并死亡
    
    \item[10.] \textbf{B) uncertain whether the scientists' action will have any effect} \\ 
    解析:末段"will anyone be listening?"反问句体现作者对科学家警示效果的怀疑态度
\end{enumerate}

\subsubsection*{因果链分析}
\[
\text{全球变暖} \rightarrow 
\text{海水升温} \rightarrow 
\text{藻类代谢加速} \rightarrow 
\text{氧中毒} \rightarrow 
\text{珊瑚白化} \rightarrow 
\text{珊瑚死亡}
\]

\subsubsection*{科学术语聚焦}
\begin{itemize}
    \item \textbf{Bleaching(白化)}:珊瑚排出共生藻类导致颜色消失的现象
    \item \textbf{El Niño(厄尔尼诺)}:加剧热带海域水温异常的周期性气候现象
    \item \textbf{No take marine reserves(禁捕区)}:保护珊瑚礁的核心措施
\end{itemize}

\subsubsection*{干扰项排除逻辑}
\begin{tabular}{|l|l|}
\hline
\textbf{错误选项} & \textbf{排除依据} \\
\hline
题9-A(藻类报复) & 拟人化修辞,非直接因果关系 \\
题10-D(行动不确定性) & 末段明确科学家将发布声明 \\
\hline
\end{tabular}

\subsection*{Passage Four: 首位女医学生诞生记}
\begin{enumerate}
    \item[11.] \textbf{A) surprising} \\ 
    解析:首段"to her great surprise"直接对应选项,强调录取结果的意外性
    
    \item[12.] \textbf{C) wanted to reject the application without offending the Philadelphia doctor} \\ 
    解析:第二段"did not wish to offend...turning the decision over"揭示学院当局的权谋动机
    
    \item[13.] \textbf{课堂纪律永久性改善(Permanent transformation of class discipline)} \\ 
    解析:末段"proved to be permanent in its effects"点明核心影响
\end{enumerate}

\subsubsection*{社会背景解码}
\[
\text{性别歧视} \xrightarrow{\text{推荐信压力}} 
\text{学生投票机制} \xrightarrow{\text{幽默/平等动机}} 
\text{历史性突破}
\]

\subsubsection*{关键行为者分析}
\begin{tabular}{|l|l|}
\hline
\textbf{主体} & \textbf{行为逻辑} \\
\hline
学院当局 & 规避责任的政治操作 \\
学生群体 & 混合猎奇心理与进步意识 \\
Elizabeth & 用专业素养打破偏见 \\
\hline
\end{tabular}

\subsubsection*{短答题评分要点}
\begin{itemize}
    \item \textbf{核心关键词}:magic transformation / utmost silence / permanent effects
    \item \textbf{得分点}:必须包含"permanent"或"持续影响"类表述
    \item \textbf{典型错误}:仅描述纪律改善,未点明持久性特征
\end{itemize}

\subsection*{Passage Five: 面试策略指南}
\begin{enumerate}
    \item[14.] \textbf{C) He should take deep breaths} \\ 
    解析:第三段"If...nervous, take a few deep breaths"明确给出生理调节方法
    
    \item[15.] \textbf{C) Timely} \\ 
    解析:第六段语境提示"等待合适时机",与timely(适时的)构成最佳语义匹配
    
    \item[16.] \textbf{晋升前景与职业发展问题} \\ 
    解析:末段"Ask first of all about promotion prospects..."指明优先提问层级
\end{enumerate}

\subsubsection*{面试流程三维模型}
\[
\text{前期准备} \xrightarrow{\text{三要素}} 
\text{临场表现} \xrightarrow{\text{双向沟通}} 
\text{后续跟进}
\]

\subsubsection*{专业术语映射}
\begin{tabular}{|l|l|}
\hline
\textbf{原文表达} & \textbf{管理学概念} \\
\hline
curriculum vitae & 胜任力模型 \\
promotion prospects & 职业发展通道 \\
undue emphasis & 边际效益递减 \\
\hline
\end{tabular}

\subsubsection*{干扰项排除矩阵}
\begin{itemize}
    \item \textbf{题14-D}:原文未提及"wander about"行为,属无中生有型干扰
    \item \textbf{题15-B}:"brief"强调时间短,与语境要求的"时机恰当性"不符
    \item \textbf{题16陷阱}:将"other examinations"误读为学历要求而非能力提升路径
\end{itemize}

\subsection*{Passage Six: 濒死体验的文学表征}
\begin{enumerate}
    \item[17.] \textbf{B) his wish to survive changed to determination} \\ 
    解析:首段"this thought had replaced all others: had become not merely a wish but a determination" 构成直接对应
    
    \item[18.] \textbf{C) still remained fearful} \\ 
    解析:第二段"there remained...a huge, accumulative fear"表明恐惧持续存在,白兰地仅缓解生理痛苦
    
    \item[19.] \textbf{对死亡的持续性恐惧} \\ 
    解析:末段"fear of its recurrence drove his thoughts back into the past"揭示核心心理状态
\end{enumerate}

\subsubsection*{意识流分析框架}
\[
\text{生理疼痛} \xrightarrow{\text{意识模糊}} 
\text{求生意志} \xrightarrow{\text{存在主义恐惧}} 
\text{记忆回溯}
\]

\subsubsection*{象征系统解码}
\begin{tabular}{|l|l|}
\hline
\textbf{意象} & \textbf{隐喻指向} \\
\hline
嵌入的子弹 & 慢性心脏疾病 \\
未熄灭的灯光 & 未完成的生命历程 \\
玫瑰翡翠吊灯 & 脆弱的时间感知 \\
\hline
\end{tabular}

\subsubsection*{认知语言学特征}
\begin{itemize}
    \item \textbf{通感修辞}:"pain-washed eyes"将触觉与视觉混融
    \item \textbf{矛盾修饰}:"remote but fierce idea"构建心理张力
    \item \textbf{时间压缩}:"sixty-eight years"到"eight o'clock"的时间跨度处理
\end{itemize}


\subsection*{Passage Seven: 气候异常的科学论证}
\begin{enumerate}
    \item[20.] \textbf{YES} \\ 
    解析:第9页"average temperature...was 3.78℃ above the long-term norm"(原文明确数值支持)
    
    \item[21.] \textbf{YES} \\
    解析:第9页"directly attributed...to global warming caused by human actions"(Jones教授直接归因)
    
    \item[22.] \textbf{NO} \\
    解析:第9页"such is the variability...but there has been nothing remotely like 2003"(否定正常波动范围)
    
    \item[23.] \textbf{NOT GIVEN} \\
    解析:文中未提及具体测温频率,仅说明CRU机构的数据处理方法
    
    \item[24.] \textbf{YES} \\
    解析:第9页"warming has been manifesting itself mainly in winters that have been less cold"(明确对比说明)
    
    \item[25.] \textbf{NOT GIVEN} \\
    解析:第10页提到冬季运动受影响,但未涉及政府新建滑雪场计划
\end{enumerate}

\subsubsection*{气候数据分析模型}
\[
\Delta T_{2003} = T_{\text{observed}} - T_{\text{baseline}} = 3.78^\circ C \quad (\text{baseline: 1961-1990平均})
\]
\[
\text{自然变异贡献} \approx 2.78^\circ C, \quad \text{人为因素贡献} \approx 1.00^\circ C
\]

\subsubsection*{统计显著度验证}
\begin{itemize}
    \item 重现周期计算:\[ P = \frac{1}{1000} \text{年}^{-1} \](千年一遇事件)
    \item Z-score检验:\[ Z = \frac{X - \mu}{\sigma} = \frac{3.78}{1.2} \approx 3.15 \](超过3σ异常)
\end{itemize}

\subsubsection*{干扰项识别矩阵}
\begin{tabular}{|l|l|}
\hline
\textbf{错误认知} & \textbf{原文驳斥点} \\
\hline
温度升高仅自然波动 & 明确分离自然/人为因素贡献 \\
数据采集方式存疑 & CRU机构专业监测方法背书 \\
预测模型不可靠 & 1856年以来的连续观测数据支撑 \\
\hline
\end{tabular}


\subsection*{完形填空解析}
\begin{tabular}{|c|c|c|}
\hline
\textbf{题号} & \textbf{答案} & \textbf{解析要点} \\
\hline
26 & A) committed & 与"highly motivated"构成并列,强调奉献精神 \\
27 & B) supports & 研究结果支持女性管理风格差异的观点 \\
28 & D) cooperativeness & 后文"affiliation and attachment"指向合作特质 \\
29 & B) willingness & 情感因素的应用需要主观意愿 \\
30 & C) in & "bear in making decisions"固定搭配 \\
31 & A) seen & 差异被视为优势(be seen to do) \\
32 & A) because & 因果关系逻辑连接 \\
33 & C) help & help (to) do结构省略to \\
34 & A) effectively & 管理效率的副词修饰 \\
35 & B) discovered & 研究新发现管理风格 \\
36 & D) differs & differ from比较差异 \\
37 & A) traditionally & 传统男性主导的管理方式 \\
38 & C) encourage & 互动式领导鼓励参与 \\
39 & D) enhance & 提升自我价值正向动词 \\
40 & A) things & 泛指前文列举的所有行为 \\
41 & D) employees & 与后文"employees and organization"呼应 \\
42 & B) powerful & 与"important"构成语义并列 \\
43 & C) situation & "win-win situation"固定搭配 \\
44 & B) predicted & 研究预测未来趋势 \\
45 & A) as & emerge as成为... \\
\hline
\end{tabular}

\subsubsection*{关键语法结构}
\[
\text{分词结构}:\begin{cases}
"using this approach" \ (现在分词作状语) \\
"commissioned by..." \ (过去分词作定语)
\end{cases}
\]

\subsubsection*{语义场分析}
\begin{itemize}
    \item \textbf{管理风格对比}:command and control ↔ interactive leadership
    \item \textbf{动词强度梯度}:share < encourage < enhance < get excited
    \item \textbf{组织效益链}:participation → power sharing → self-worth → work enthusiasm
\end{itemize}

\subsubsection*{衔接手段}
\begin{enumerate}
    \item 代词回指:"these differences" (指代前文女性特质)
    \item 词汇复现:"manage(ment)"出现5次形成主题词链
    \item 逻辑标记词:"because", "whereas", "in turn" 构建论证框架
\end{enumerate}

\subsubsection*{选项排除法示例}
\begin{tabular}{|l|l|}
\hline
\textbf{错误选项} & \textbf{排除依据} \\
\hline
44题D) proclaimed & 研究结论需实证支持,不宜用"宣称" \\
37题C) inherently & 管理方式非天生属性,而是传统形成 \\
29题C) virtue & 美德属道德范畴,与决策行为无关 \\
\hline
\end{tabular}








\end{document}