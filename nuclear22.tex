\documentclass{article}
\usepackage{amsmath}
\usepackage{siunitx}
\usepackage{ctex}
\begin{document}

\section*{一、名词解释(15 分,每题 3 分)}
\begin{enumerate}
    \item \textbf{衰变常量}:放射性核素在单位时间内发生衰变的概率,用$\lambda$表示,单位是$\mathrm{s^{-1}}$。
    
    \item \textbf{原子核的同位旋}:描述原子核中质子与中子对称性的量子数,表征核子的电荷独立性。
    
    \item \textbf{镜像核}:质子数和中子数互换的一对原子核(如$^{13}\mathrm{C}$和$^{13}\mathrm{N}$)。
    
    \item \textbf{内转换现象}:原子核退激时直接将能量传递给核外电子使其电离的现象(非γ辐射)。
    
    \item \textbf{渡越时间}:p78带电粒子在介质中从获得能量到损失全部能量所经历的时间。
\end{enumerate}

\section*{二、简述题(25 分,每题 5 分)}
\begin{enumerate}
    \item \textbf{高速正电子作用过程}:p62
    \begin{itemize}
        \item 电离损失:与原子电子发生库仑作用损失能量
        \item 辐射损失:产生轫致辐射
        \item 湮灭:与电子发生湮灭产生双$\gamma$光子(511 keV)
    \end{itemize}
    
    \item \textbf{核反应三阶段及守恒律}:
    三阶段模型:
    \begin{enumerate}
        \item 独立粒子阶段(入射粒子与靶核作用)
        \item 复合系统阶段(形成中间复合核)
        \item 衰变阶段(复合核衰变)
    \end{enumerate}
    守恒定律:能量、动量、角动量、电荷、重子数
    
    \item \textbf{降低源外本底措施}:pp346
    \begin{itemize}
        \item 屏蔽:铅室、铜屏蔽层
        \item 符合测量:反符合计数器
        \item 低温运行:减少热噪声
        \item 材料筛选:低本底材料
    \end{itemize}
    
    \item \textbf{中子探测原理}:pp297
    \begin{itemize}
        \item 核反应法:利用中子引发的核反应
        \item 典型反应:$^{10}\mathrm{B}(n,\alpha)^{7}\mathrm{Li}$,$^{3}\mathrm{He}(n,p)^{3}\mathrm{H}$
        \item 探测器:BF$_3$正比计数器、锂玻璃探测器
    \end{itemize}
    
    \item \textbf{β能谱特点及中微子解释}:
    特点:
    \begin{itemize}
        \item 连续能谱
        \item 存在最大能量$E_{max}$
        \item 平均能量约为$E_{max}/3$
    \end{itemize}
    中微子假说:β衰变时能量在三体(β粒子、中微子、子核)间分配,解释连续谱和能量守恒
\end{enumerate}

\section*{三、计算题(60 分,每题 15 分)}
\begin{enumerate}
    \item \textbf{核反应产额计算}:
    
    \textbf{已知}:
    \begin{align*}
        I &= 1\ \mathrm{nA} = 1\times10^{-9}\ \mathrm{C/s} \\
        \sigma &= 450\ \mathrm{mb} = 450\times10^{-27}\ \mathrm{cm^2} \\
        \text{靶厚} &= 1\ \mathrm{mg/cm^2}
    \end{align*}
    
    \textbf{解题步骤}:
    \begin{enumerate}
        \item 单位时间内入射粒子总数:
        $$
        N_{beam} = \frac{I}{e} = \frac{1\times10^{-9}}{1.6\times10^{-19}} = 6.25\times10^{9}\ \mathrm{s^{-1}}
        $$
        
        \item 计算靶原子面密度:
        $$
        N_{s} = \frac{N_A D}{A} = \frac{1\times10^{-3}}{67}\times6.022\times10^{23} = 8.99\times10^{18}\ \mathrm{cm^{-2}}
        $$

        \item 计算产额:
        $$
        Y = N_{s} \cdot \sigma
        $$
        
        \item 单位时间生成的复合核数:
        $$
        N = N_{beam} \cdot Y = 6.25\times10^{9} \times 8.99\times10^{18} \times 450\times10^{-27} = 2.53 \times10^{4}\ \mathrm{s^{-1}}
        $$
    \end{enumerate}
    
    \item \textbf{地球年龄计算}:
    
    \textbf{已知}:
    \begin{align*}
        T_{235} &= 7.038\times10^8\ \mathrm{年} \\
        T_{238} &= 4.468\times10^9\ \mathrm{年} \\
        \frac{N_{235}}{N_{238}}|_{现在} &= \frac{0.72}{99.2745} \\
        \frac{N_{235}}{N_{238}}|_{初始} &= \frac{1}{2}
    \end{align*}
    
    \textbf{解题步骤}:
    \begin{enumerate}
        \item 建立衰变方程:
        $$
        \frac{N_{235}}{N_{238}} = \frac{1}{2} e^{-(\lambda_{235}-\lambda_{238})t}
        $$
        
        \item 代入当前比例:
        $$
        \frac{0.72}{99.2745} = \frac{1}{2} e^{-(\frac{\ln2}{7.038\times10^8} - \frac{\ln2}{4.468\times10^9})t}
        $$
        
        \item 解方程得:
        $$
        t \approx \frac{\ln(2 \times 0.72/99.2745)}{\frac{1}{4.468\times10^9} - \frac{1}{7.038\times10^8}} \approx 6.02\times10^9\ \mathrm{年}
        $$
    \end{enumerate}
\end{enumerate}

\section*{题目3}
计算$\mathrm{^{24}Na}$的2.76MeV $\gamma$射线在NaI(Tl)单晶谱仪的输出脉冲幅度谱上,康普顿边缘与单逃逸峰之间的相对位置。

\subsection*{解答p66}

\paragraph{步骤1:计算康普顿边缘能量}
康普顿边缘对应最大反冲电子能量。根据康普顿边缘公式:
$$
E_{emax} =   \frac{E_\gamma}{1 + \frac{m_0 c^2}{2E_\gamma}} 
$$
其中,$E_\gamma = \SI{2.76}{MeV}$,$m_e c^2 = \SI{0.511}{MeV}$。代入得:
$$
E_{emax} = \SI{2.526}{MeV}
$$

\paragraph{步骤2:计算单逃逸峰能量}
单逃逸峰由电子对效应产生,逃逸一个$\SI{0.511}{MeV}$光子:
$$
E_{\text{单逃逸}} = E_\gamma - 0.511 = \SI{2.249}{MeV}
$$

\paragraph{步骤3:确定相对位置}
在能谱中,能量与脉冲幅度成正比。康普顿边缘($\SI{2.526}{MeV}$)对应的幅度高于单逃逸峰($\SI{2.249}{MeV}$),因此康普顿边缘位于单逃逸峰的\textbf{右侧}。

\section*{题目4}
在测量极化度的实验中,一般在相对束流方向的两侧同样角度安排两个相同的探测器。设$R$和$L$分别为右侧和左侧探测器的总计数(假设无本底),极化度定义为:
$$
P = \frac{R - L}{R + L}
$$
试以$R$和$L$为自变量,推导$P$的标准误差表达式。

\subsection*{解答}

\paragraph{步骤1:误差传播公式}
$P$的标准误差$\sigma_P$由$R$和$L$的方差决定:
$$
\sigma_P = \sqrt{ \left( \frac{\partial P}{\partial R} \right)^2 \sigma_R^2 + \left( \frac{\partial P}{\partial L} \right)^2 \sigma_L^2 }
$$

\paragraph{步骤2:计算偏导数}
$$
\frac{\partial P}{\partial R} = \frac{(R+L) - (R-L)}{(R+L)^2} = \frac{2L}{(R+L)^2}, \quad
\frac{\partial P}{\partial L} = \frac{-(R+L) - (R-L)}{(R+L)^2} = \frac{-2R}{(R+L)^2}
$$

\paragraph{步骤3:泊松分布假设}
计数服从泊松分布,$\sigma_R^2 = R$, $\sigma_L^2 = L$,代入得:
$$
\sigma_P = \sqrt{ \left( \frac{2L}{(R+L)^2} \right)^2 R + \left( \frac{-2R}{(R+L)^2} \right)^2 L }
$$

\paragraph{步骤4:简化表达式}
$$
\sigma_P = \frac{2}{(R+L)^2} \sqrt{L^2 R + R^2 L} = \frac{2\sqrt{RL(R+L)}}{(R+L)^2} = \frac{2\sqrt{RL}}{(R+L)^{3/2}}
$$

\paragraph{最终表达式}
$$
\sigma_P = \frac{2\sqrt{RL}}{(R+L)^{3/2}} = \frac{2\sqrt{RL}}{(R+L)\sqrt{R+L}}
$$
\end{document}