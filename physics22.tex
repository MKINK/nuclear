\documentclass{article}
\usepackage{amsmath}
\usepackage{physics}
\usepackage{enumitem}
\usepackage{esint}
\usepackage{ctex}  
\title{物理试题与解答}
\author{}
\date{}

\begin{document}

\maketitle

\section*{一、简答题(25分,每小题5分)}
\begin{enumerate}[label=\arabic*.]
    \item 圆锥摆的摆球在水平面内作圆周运动:
    \begin{itemize}
        \item 动量是否守恒?角动量是否守恒?
        \item 选择摆球与地球组成的系统,机械能是否守恒?
    \end{itemize}
    \textbf{解答:}
    \begin{itemize}
        \item 动量不守恒:摆球受绳张力作用,存在外力 $\displaystyle \sum \vec{F} = \frac{mv^2}{r} \neq 0$
        \item 角动量守恒:合外力矩 $\displaystyle \sum \vec{M} = \vec{r} \times \vec{T} \equiv 0$(张力始终指向圆心)
        \item 机械能守恒:系统仅受保守力(重力)做功,$\Delta E_{mec} = W_{nc} = 0$
    \end{itemize}

    \item 为什么热力学过程中的功仅与初末态有关,而与中间过程无关?\\
    \textbf{解答:} 题目表述有误,准确应为绝热过程。根据热力学第一定律:
    $$
    \Delta U = Q - W
    $$
    绝热过程$Q=0$,故$W = -\Delta U$。因内能是状态量,$\Delta U$仅取决于初末态,故绝热功与路径无关。

    \item 孤立导体球壳中心放置点电荷时,内外表面电荷分布是否均匀?若点电荷偏离球心,情况如何?\\
    \textbf{解答:}
    \begin{itemize}
        \item 中心放置:内表面均匀($\oint \vec{E} \cdot d\vec{a} = Q_{enc}/\varepsilon_0$),外表面均匀(静电平衡导体表面等势)
        \item 偏离球心:内表面非均匀(感应电荷分布不对称),外表面仍均匀(外部电场屏蔽)
    \end{itemize}

    \item 线偏振光通过偏振片后振动方向转过90度,至少需要几块理想偏振片?此时透射光强最大值是原光强的几倍?\\
    \textbf{解答:} 设初始偏振方向与最终方向夹角为$\theta$,由马吕斯定律:
    $$
    I = I_0 \cos^2\theta
    $$
    单次旋转90°时$I=0$,故需至少2片。设中间片转$\alpha$角:
    $$
    I = I_0 \cos^2\alpha \cdot \cos^2(90^\circ-\alpha) = \frac{I_0}{4}\sin^2(2\alpha)
    $$
    当$\alpha=45^\circ$时取得最大值$I_{max} = I_0/4$

    \item 碱金属原子能级与轨道角量子数相关的原因是什么?造成精细能级的原因是什么?\\
    \textbf{解答:}
    \begin{itemize}
        \item 主因:原子实的极化与轨道贯穿效应(不同$l$值的电子受原子实屏蔽不同)
        \item 精细结构:相对论修正(自旋-轨道耦合作用)$\Delta E = \frac{\alpha^2}{n^3}\left( \frac{1}{j+1/2} \right)$
    \end{itemize}
\end{enumerate}

\section*{二、计算题}
\begin{enumerate}[label=\arabic*.]
    \item (20分)半径为$R$的接地金属球外包裹相对介电常数为$\varepsilon_r$、外径$2R$的电介质壳,介质内均匀分布自由电荷$q_0$,求介质壳外表面电势。
    
    \textbf{解答:}
    \begin{enumerate}
        \item 自由电荷体密度:$\rho = \dfrac{3q_0}{4\pi[(2R)^3 - R^3]} = \dfrac{3q_0}{28\pi R^3}$
        \item 极化电荷:$\rho_p = -\nabla \cdot \vec{P} = -\dfrac{\varepsilon_r-1}{\varepsilon_r}\rho$
        \item 金属球感应电荷$q'$,由接地条件$\phi_{total}(R) = 0$:
        $$
        \dfrac{1}{4\pi\varepsilon_0}\left( \dfrac{q'}{R} + \dfrac{q_0 + q'}{2R\varepsilon_r} \right) = 0 \Rightarrow q' = -\dfrac{q_0}{1+2\varepsilon_r}
        $$
        \item 外表面电势:
        $$
        \phi = \dfrac{q_0 + q'}{4\pi\varepsilon_0(2R)} = \dfrac{q_0}{8\pi\varepsilon_0 R}\left(1 - \dfrac{1}{\varepsilon_r}\right)
        $$
    \end{enumerate}

    \item (15分)圆形极板电容器(极板面积$S$,间距$d$)通过细导线(电阻$R$,长$d$)连接交变电压$U=U_0\sin\omega t$,求:
    \begin{enumerate}
        \item 细导线电流
        \item 位移电流
        \item 外接线电流
        \item 极板间距轴线$r$处($r<a$)的磁场强度
    \end{enumerate}
    
    \textbf{解答:}
    \begin{enumerate}
        \item 传导电流:
        $$
        I_c = \dfrac{U}{R} = \dfrac{U_0}{R}\sin\omega t
        $$
        \item 位移电流:
        $$
        I_D = \varepsilon_0 \dfrac{d\Phi_E}{dt} = \varepsilon_0 \dfrac{d}{dt}\left( \dfrac{U}{d}S \right) = \dfrac{\varepsilon_0 S U_0 \omega}{d}\cos\omega t
        $$
        \item 外接线电流(全电流连续性):
        $$
        I_{total} = I_c + I_D = \dfrac{U_0}{R}\sin\omega t + \dfrac{\varepsilon_0 S U_0 \omega}{d}\cos\omega t
        $$
        \item 由安培环路定理修正形式:
        $$
        \oint \vec{H} \cdot d\vec{l} = I_{enc} + \dfrac{d}{dt}\int \vec{D} \cdot d\vec{a}
        $$
        取半径$r$的环路:
        $$
        H \cdot 2\pi r = \dfrac{\pi r^2}{\pi a^2}I_{total} \Rightarrow H = \dfrac{r}{2\pi a^2}\left( \dfrac{U_0}{R}\sin\omega t + \dfrac{\varepsilon_0 S U_0 \omega}{d}\cos\omega t \right)
        $$
    \end{enumerate}

    \item (20分)单原子理想气体进行x过程,其$p\text{-}V$曲线下移$p_0$后与$T_0$等温线重合,求:
    \begin{enumerate}
        \item x过程的$V\text{-}T$关系式
        \item 比热$c$与压强$p$的关系
    \end{enumerate}
    
    \textbf{解答:}
    \begin{enumerate}
        \item 由题意得x过程方程:
        $$
        p = \dfrac{\nu R T}{V} - p_0
        $$
        整理得$V\text{-}T$关系:
        $$
        V = \dfrac{\nu R T}{p + p_0}
        $$
        \item 热力学第一定律:
        $$
        dQ = \dfrac{3}{2}\nu R dT + p dV
        $$
        将$dV = \dfrac{\nu R dT}{p + p_0} - \dfrac{\nu R T}{(p + p_0)^2}dp$代入:
        $$
        c = \dfrac{1}{\nu}\dfrac{dQ}{dT} = \dfrac{3R}{2} + \dfrac{p R}{p + p_0} - \dfrac{p R T}{(p + p_0)^2}\dfrac{dp}{dT}
        $$
        最终化简得:
        $$
        c = \dfrac{5R}{2} - \dfrac{R p_0}{p}
        $$
    \end{enumerate}

    \item (20分)杨氏干涉装置中用三缝(缝距$d$和$\frac{3d}{2}$)替代双缝,求:
    \begin{enumerate}
        \item 幕上强度分布公式
        \item 一级极大角位置$\theta$
        \item $\theta=\frac{\pi}{2}$方向强度与一级极大强度之比
    \end{enumerate}
    
    \textbf{解答:错误}
    \begin{enumerate}
        \item 三缝干涉振幅叠加:
        $$
        A = A_0 \left[ 1 + e^{i\delta_1} + e^{i\delta_2} \right]
        $$
        其中$\delta_1 = \dfrac{2\pi d \sin\theta}{\lambda}$, $\delta_2 = \dfrac{3\pi d \sin\theta}{\lambda}$,强度分布:
        $$
        I = I_0 \left[ 3 + 4\cos\delta_1 + 2\cos(\delta_2 - \delta_1) \right]
        $$
        化简得:
        $$
        I(\theta) = I_0 \left[ 3 + 4\cos\left(\dfrac{\pi d \sin\theta}{\lambda}\right) + 2\cos\left(\dfrac{3\pi d \sin\theta}{\lambda}\right) \right]
        $$
        \item 一级极大条件:
        $$
        \dfrac{\pi d \sin\theta}{\lambda} = \pm \pi \Rightarrow \theta = \arcsin\left( \dfrac{\lambda}{3d} \right)
        $$
        \item 当$\theta = \pi/2$时:
        $$
        I_{\pi/2} = I_0 [3 + 4\cos(\pi/2) + 2\cos(3\pi/2)] = 3I_0
        $$
        一级极大强度$I_{max} = 9I_0$,故比值:
        $$
        \dfrac{I_{\pi/2}}{I_{max}} = \dfrac{1}{3}
        $$
    \end{enumerate}
\end{enumerate}

介质壳外表面的电势计算如下:

1. **电荷分布与电场计算**:  
   电介质壳内(\( R \leq r \leq 2R \))的自由电荷体密度为 \(\rho = \frac{3q_0}{28\pi R^3}\)。通过高斯定理,电位移矢量 \( D(r) \) 为:  
   \[
   D(r) = \frac{q_0 (r^3 - R^3)}{28\pi R^3 r^2}
   \]  
   对应的电场强度 \( E_1(r) = \frac{D(r)}{\varepsilon_0 \varepsilon_r} = \frac{q_0 (r^3 - R^3)}{28\pi \varepsilon_0 \varepsilon_r R^3 r^2} \)。

2. **金属球电荷确定**:  
   金属球接地(电势为零),设其电荷为 \( Q \)。总电势由电介质壳内外的电场积分决定:  
   \[
   \int_R^{2R} E_1(r) \, dr + \int_{2R}^\infty E_2(r) \, dr = 0
   \]  
   解得 \( Q = -\frac{q_0 (7\varepsilon_r + 2)}{7\varepsilon_r} \)。

3. **介质壳外电场与电势**:  
   外部区域(\( r \geq 2R \))的电场为:  
   \[
   E_2(r) = \frac{Q + q_0}{4\pi \varepsilon_0 r^2}
   \]  
   代入 \( Q + q_0 = -\frac{2q_0}{7\varepsilon_r} \),积分得到介质壳外表面(\( r = 2R \))的电势:  
   \[
   \phi = \int_{2R}^\infty E_2(r) \, dr = -\frac{q_0}{28\pi \varepsilon_0 \varepsilon_r R}
   \]

**答案**:  
介质壳外表面的电势为  
\[
\boxed{-\dfrac{q_0}{28\pi \varepsilon_0 \varepsilon_r R}}
\]

这道题目要求计算介质壳外表面的电势,已知金属球接地,介质壳内均匀分布自由电荷 \( q_0 \),相对介电常数为 \( \varepsilon_r \)。以下是详细的解题过程:

---

### **解题步骤**

#### **1. 划分区域并分析对称性**
• **金属球内部**(\( r < R \)):导体内部电场为零,电势与接地条件一致,即 \( \phi(R) = 0 \)。
• **介质壳内部**(\( R \le r \le 2R \)):自由电荷 \( q_0 \) 均匀分布,介电常数为 \( \varepsilon_r \)。
• **介质壳外部**(\( r > 2R \)):视为真空(介电常数 \( \varepsilon_0 \))。

#### **2. 确定电位移 \( D(r) \)**
利用高斯定理,仅考虑自由电荷:
• **介质壳内部**(\( R \le r \le 2R \)):
  \[
  D(r) \cdot 4\pi r^2 = Q + \frac{q_0 (r^3 - R^3)}{7R^3},
  \]
  解得:
  \[
  D(r) = \frac{1}{4\pi r^2} \left[ Q + \frac{q_0 (r^3 - R^3)}{7R^3} \right].
  \]
• **介质壳外部**(\( r > 2R \)):
  \[
  D(r) = \frac{Q + q_0}{4\pi r^2}.
  \]

#### **3. 计算电场强度 \( E(r) \)**
• **介质壳内部**:\( E_1(r) = \frac{D(r)}{\varepsilon_0 \varepsilon_r} \).
• **介质壳外部**:\( E_2(r) = \frac{D(r)}{\varepsilon_0} \).

#### **4. 金属球接地条件**
金属球电势为零,积分电场从 \( r = R \) 到无穷远:
\[
\int_{R}^{2R} E_1 \, dr + \int_{2R}^{\infty} E_2 \, dr = 0.
\]
代入 \( E_1 \) 和 \( E_2 \),解得金属球上的电荷 \( Q \):
\[
Q = -\frac{q_0 (7\varepsilon_r + 2)}{7(\varepsilon_r + 1)}.
\]

#### **5. 计算介质壳外表面电势**
介质壳外表面(\( r = 2R \))的电势为外部电场积分:
\[
\phi(2R) = \int_{2R}^{\infty} E_2 \, dr = \frac{Q + q_0}{8\pi \varepsilon_0 R}.
\]
代入 \( Q \) 的表达式并化简:
\[
\phi(2R) = \frac{5q_0}{56\pi \varepsilon_0 R (\varepsilon_r + 1)}.
\]

---

### **最终答案**
介质壳外表面的电势为:
\[
\phi(2R) = \boxed{\frac{5q_0}{56\pi \varepsilon_0 R (\varepsilon_r + 1)}}.
\]

---

### **关键点总结**
1. **接地条件**:金属球电势为零,通过积分总电场确定其电荷 \( Q \)。
2. **高斯定理应用**:介质中仅自由电荷影响电位移 \( D \),结合介电常数计算电场。
3. **电势叠加**:介质壳外表面电势由外部电场积分得到,与 \( Q \) 和 \( q_0 \) 相关。

此解法系统性地结合了导体接地、介质极化和高斯定理,确保了电势计算的准确性。

### 解答:

**1. x过程的V-T关系式推导:**

• 题目指出,x过程曲线向下平移\( p_0 \)后与温度为\( T_0 \)的等温线重合。等温过程方程为\( pV = \nu R T_0 \),因此平移后的x过程方程为:
  \[
  p - p_0 = \frac{\nu R T_0}{V} \quad \Rightarrow \quad p = \frac{\nu R T_0}{V} + p_0
  \]
  
• 结合理想气体状态方程\( pV = \nu R T \),代入上式得:
  \[
  \left( \frac{\nu R T_0}{V} + p_0 \right) V = \nu R T \quad \Rightarrow \quad \nu R T_0 + p_0 V = \nu R T
  \]
  
• 整理得V-T关系式:
  \[
  V = \frac{\nu R}{p_0} (T - T_0)
  \]

**2. x过程的比热\( c \)与压强\( p \)的关系推导:**

• 热力学第一定律:\( dQ = \nu c dT = \nu C_v dT + p dV \),其中单原子气体\( C_v = \frac{3}{2}R \)。

• 利用已求出的\( V = \frac{\nu R}{p_0} (T - T_0) \),对温度求导得:
  \[
  \frac{dV}{dT} = \frac{\nu R}{p_0}
  \]

• 代入热力学第一定律,消去\( \nu dT \)得:
  \[
  c = C_v + \frac{p}{\nu} \cdot \frac{dV}{dT} = \frac{3R}{2} + \frac{p R}{p_0}
  \]

### 最终答案:

1. **x过程的V-T关系式**为:
   \[
   V = \frac{v R}{p_0} (T - T_0)
   \]
   (其中\( v \)为摩尔数,题目中符号为v)

2. **x过程的比热\( c \)与压强\( p \)的关系**为:
   \[
   c = \frac{3R}{2} + \frac{R}{p_0} p
   \]

---

**注:** 答案已根据题目符号调整,确保与题目中使用的物理量符号一致(如摩尔数用\( v \)而非\( \nu \))。

\end{document}